% Lectures on Computer Architecture
% Compile with pdfLaTeX
\documentclass[11pt, a4paper, oneside]{book}

% ---------- FONTS & ENCODING ----------
\usepackage[utf8]{inputenc}
\usepackage[T1]{fontenc}
\usepackage{lmodern}  % Latin Modern fonts - better than CM

% ---------- LAYOUT ----------
\usepackage[paper=a4paper, margin=2.5cm, top=3.5cm, bottom=3.5cm, headheight=14pt]{geometry}
\usepackage{ragged2e}
\usepackage{setspace}
\setstretch{1.1}  % Slightly increased line spacing for readability

% ---------- ESSENTIAL PACKAGES ----------
\usepackage{graphicx}
\usepackage{amsmath}
\usepackage{amssymb}
\usepackage{listings}
\usepackage{enumitem}

% Configure list formatting for consistent alignment
\setlist[itemize]{leftmargin=*, topsep=8pt, itemsep=4pt, parsep=0pt, partopsep=0pt}
\setlist[enumerate]{leftmargin=*, topsep=8pt, itemsep=4pt, parsep=0pt, partopsep=0pt}

% ---------- COLORS ----------
\usepackage{xcolor}
\definecolor{accentblue}{HTML}{1E9EEB} % tweak hex to taste (bright cyan-blue)
\definecolor{accentdark}{HTML}{0E6A8A} % darker blue/teal if needed
\definecolor{accentgray}{gray}{0.45}

% ---------- HEADER / FOOTER & BACKGROUNDS ----------
\usepackage{fancyhdr}
\usepackage{tikz}
\usepackage{eso-pic} % add to background

\pagestyle{fancy}
\fancyhf{} % clear header/footer
% page number at top-right (small)
\fancyhead[R]{\footnotesize\thepage}
\renewcommand{\headrulewidth}{0pt}
\renewcommand{\footrulewidth}{0pt}

% Decorative bars drawn on every page (full bleed to paper edge)
% Adjusted to remove interior white margins and reach trim edges.
\AddToShipoutPictureBG{%
  \begin{tikzpicture}[remember picture,overlay]
    % Top bar: full width flush with top edge
    \fill [accentblue] (current page.north west) rectangle ([yshift=-0.30cm] current page.north east);
    % Thin separator line below top bar
    % Bottom bar enlarged (from 0.55cm to 0.90cm)
    \fill [accentblue] (current page.south west) rectangle ([yshift=1.20cm] current page.south east);
    % Thin separator line above enlarged bottom bar
  \end{tikzpicture}%
}

% To avoid the decorative bars covering content, push text away from top/bottom:
% Reduce head/foot separation to reflect thinner bars now
\setlength{\headsep}{20pt}
\setlength{\footskip}{30pt}

% ---------- TITLE / COVER STYLING ----------
\usepackage{titling}
\pretitle{\vspace*{4cm}\raggedright} % push title down similar to sample
\posttitle{\par\vspace{0.5em}}
\preauthor{\vspace{2em}\raggedright\small}
\postauthor{\par\vspace{0.5em}}
\predate{}\postdate{}

% ---------- SECTION / HEADING STYLES ----------
\usepackage{titlesec}
\titleformat{\chapter}[display]
  {\normalfont\Huge\bfseries\sffamily\color{accentblue}}
  {\chaptertitlename\ \thechapter}{20pt}{\Huge}
\titlespacing*{\chapter}{0pt}{0pt}{30pt}

\titleformat{\section}
  {\Large\bfseries\sffamily\color{accentblue}}
  {\thesection}{1em}{}
\titlespacing*{\section}{0pt}{3.5ex plus 1ex minus .2ex}{2.3ex plus .2ex}

\titleformat{\subsection}
  {\large\bfseries\sffamily\color{accentdark}}
  {\thesubsection}{1em}{}
\titlespacing*{\subsection}{0pt}{3.25ex plus 1ex minus .2ex}{1.5ex plus .2ex}

\titleformat{\subsubsection}
  {\normalsize\bfseries\sffamily\color{accentdark}}
  {\thesubsubsection}{1em}{}
\titlespacing*{\subsubsection}{0pt}{3.25ex plus 1ex minus .2ex}{1.5ex plus .2ex}

\titleformat{\paragraph}[runin]
  {\normalsize\bfseries\color{black}}
  {}{0em}{}
\titlespacing*{\paragraph}{0pt}{2ex plus 1ex minus .2ex}{1em}

% ---------- TABLE OF CONTENTS STYLING ----------
\usepackage{tocloft}
\renewcommand{\cftsecfont}{\sffamily\small\color{accentblue}}
\renewcommand{\cftsecpagefont}{\sffamily\small\color{accentblue}}
\renewcommand{\cftbeforesecskip}{4pt}
\renewcommand{\contentsname}{\color{accentblue}\Large\bfseries Content}

% ---------- CODE LISTINGS ----------
\lstset{
    basicstyle=\ttfamily\small,
    breaklines=true,
    frame=single,
    numbers=left,
    numberstyle=\tiny\color{gray},
    tabsize=4,
    backgroundcolor=\color{gray!10},
    commentstyle=\color{green!60!black},
    keywordstyle=\color{blue},
    stringstyle=\color{red!70!black}
}

% ---------- HYPERREF AND URL ----------
\usepackage{url}
\usepackage{hyperref}
\hypersetup{
  colorlinks=true,
  linkcolor=accentdark,
  urlcolor=accentblue,
  citecolor=accentblue,
  pdftitle={Lectures on Computer Architecture},
  pdfauthor={Dr. Isuru Nawinne},
  bookmarksnumbered=true,
  bookmarksopen=true
}

% ---------- DOCUMENT ----------
\begin{document}

% ----------------- COVER PAGE -----------------
\begin{titlepage}
  \thispagestyle{empty}
  \vspace*{3cm}
  \begin{center}
  % Title
  {\Huge\bfseries Lectures on\\[0.5em] Computer Architecture\par}
  \vspace{3cm}
  % Author
  {\Large By Isuru Nawinne\par}
  \end{center}
  \vfill
\end{titlepage}

% ----------------- COPYRIGHT PAGE -----------------
\clearpage
\thispagestyle{empty}
\vspace*{1cm}

\begin{center}
{\Large\bfseries Lectures on Computer Architecture}\par
\vspace{0.5cm}
{\large By Isuru Nawinne}\par
\end{center}

\vspace{1.5cm}

\noindent\textcopyright\ 2025, by Creative Commons. This work is licensed under a Creative Commons Attribution-NonCommercial 4.0 International. To view a copy of this license, visit \url{https://creativecommons.org/licenses/by-nc/4.0/} or send a letter to Creative Commons, PO Box 1866, Mountain View, CA 94042, USA. This license allows reusers to distribute, remix, adapt, and build upon the material in any medium or format, so long as attribution is given to the creator. The license allows only for non-commercial use.\par

\vspace{0.8cm}

\noindent\textbf{ISBN} 978-624-92913-0-0\par

\vspace{0.8cm}

\noindent Downloadable ebook and supplementary material available at\par
\noindent\url{https://cepdnaclk.github.io/Computer-Architecture-Web}\par

\vspace{0.8cm}

\noindent\textbf{Publisher:}\par
\noindent Dr. Isuru Nawinne,\par
\noindent Department of Computer Engineering, Faculty of Engineering,\par
\noindent University of Peradeniya,\par
\noindent Peradeniya 20400,\par
\noindent Sri Lanka.\par
\noindent\href{mailto:isurunawinne@eng.pdn.ac.lk}{isurunawinne@eng.pdn.ac.lk} \url{https://people.ce.pdn.ac.lk/staff/academic/isuru-nawinne/}\par

\cleardoublepage
\pagestyle{fancy}

% Set page numbering for front matter
\pagenumbering{arabic}

% ----------------- FRONT MATTER -----------------
\tableofcontents
\clearpage

% ----------------- PREFACE -----------------
\section*{Preface}
\addcontentsline{toc}{section}{Preface}

Computer architecture sits at the heart of modern computing. It is the discipline that reveals how machines execute instructions, manage data, and achieve programmability—bridging the conceptual world of algorithms and the physical realities of hardware. Over many years of teaching this subject to undergraduate students, I have found that curiosity grows not only from understanding what computers do, but from discovering how they do it and why they're designed that way.\par

\vspace{0.5cm}

This book, \emph{Lectures on Computer Architecture}, is built upon the lecture series delivered to undergraduate cohorts. Each section distills core ideas, clarifies subtle concepts, and connects theory to the practical systems. The video lectures grew from classroom sessions, refined through questions, discussions, and repeated teaching experience. The accompanying notes are designed to complement the videos rather than duplicate it, offering multiple modalities through which students can explore the subject.\par

\vspace{0.5cm}

My goal is to provide a learning resource that is rigorous yet approachable, structured yet flexible, and suitable for both guided instruction and independent study. Whether used as a primary course text, or a guide for self-study, I hope this book supports students in creating a solid cognitive model of how computers are built.\par

\vspace{0.5cm}

I am grateful to all my students over the years whose questions and feedback helped refine these explanations, and to everyone who encouraged the development of a resource that unifies both lecture and text. It is my hope that this book helps you to see computer architecture not merely as a subject to be completed, but as a foundation for understanding the modern machines that shape our world.\par

\vspace{0.5cm}

I would like to convey my sincere appreciation to Dr. Kisaru Liyanage and Dr. Swarnalatha Radhakrishnan for their valuable contributions in delivering selected lectures. My profound thanks go to Kanishka Gunawardana and Sanka Peeris, for helping me edit this book and setting up the interactive web version. Their careful attention to detail, thoughtful feedback, and commitment to ensuring the clarity and accuracy have contributed greatly to the quality and reliability of this work.\par

\vspace{1cm}

\noindent Isuru Nawinne\par
\noindent Senior Lecturer in Computer Engineering\par

\clearpage

% ----------------- LEARNING METHODS -----------------
\section*{Learning Methods}
\addcontentsline{toc}{section}{Learning Methods}

This book is designed to support active, independent, and flexible learning, aligning closely with flipped learning and self-directed study practices.\par

\vspace{0.5cm}

The flipped-learning approach encourages students to engage with key concepts before coming to class or attempting exercises. Each section in this book includes a corresponding video lecture that introduces the fundamental ideas, explains core mechanisms, and walks through examples. Watching the video beforehand allows learners to arrive at discussions or problem-solving sessions better prepared, able to ask informed questions, and ready to dive deeper.\par

\vspace{0.5cm}

Flipped-learning transforms the role of classroom or study time: instead of passively receiving information, students actively seek and apply it. With multiple modalities of the videos as the initial exposure and the book as a reference and reinforcement tool, learners can use their interactive time: whether in discussions; tutorials; labs; or group study; to focus on reasoning, analysis, and synthesis.\par

\vspace{0.5cm}

Computer architecture is a subject that rewards curiosity and exploration. To support self-directed learning, each chapter is structured so students can progress at their own pace. The notes are carefully layered. They begin with foundational principles and incrementally build toward more advanced ideas.\par

\vspace{0.5cm}

Students are encouraged to:\par

\vspace{0.3cm}

\begin{itemize}
\item Watch the video lectures as many times as needed to internalize concepts;
\item Revisit diagrams and derivations to strengthen visual and mathematical intuition;
\item Use end-of-section summaries and conceptual checkpoints to evaluate their understanding; and
\item Make connections between topics—for example, how pipelining interacts with branching, or how memory hierarchy influences performance.
\end{itemize}

\vspace{0.5cm}

This style of learning builds autonomy, critical thinking, and long-term retention—key skills for an engineer.\par

\clearpage

% ----------------- INTRODUCTION -----------------
\section*{Introduction}
\addcontentsline{toc}{section}{Introduction}

This book follows a gradual progression from fundamental concepts to advanced architectural mechanisms, mirroring the structure of the lecture series. The material is organized into twenty sections, each corresponding to a major topic typically covered in an undergraduate computer architecture course.\par

\vspace{0.5cm}

\noindent\textbf{How the Book Is Organized}\par

\vspace{0.3cm}

The first set of chapters: Computer Abstractions, Technology Trends, and Performance establish the context and quantitative foundation needed to reason about architectural decisions. These are followed by chapters on Assembly Language Programming, Number Representation, Branching, Function Calls, and Memory Access which build the low-level understanding of how instructions operate.\par

\vspace{0.5cm}

Midway through the book, the focus shifts to the execution engine itself: Microarchitecture, Datapath, Control, and the progression from Single-Cycle Execution to Pipelined Processors. The chapters such as Pipeline Analysis help students understand real-world engineering challenges.\par

\vspace{0.5cm}

The later sections explore the memory subsystem in depth: Memory Hierarchy, Caching, Direct Mapped and Associative Cache Control, Multi-Level Caches, and Virtual Memory, before extending the architectural view to Multiprocessors, Storage, and Interfacing.\par

\vspace{0.5cm}

Each chapter includes:\par

\vspace{0.3cm}

\begin{itemize}
\item A complete video lecture that introduces and explains concepts
\item Written notes highlighting definitions, diagrams, examples, and reasoning steps
\item Clarifications of common misconceptions
\item Connections to earlier and later material
\item Guidance on how the topic relates to real processors and modern systems
\end{itemize}

\vspace{0.5cm}

\noindent\textbf{How to Use the Videos and Notes}\par

\vspace{0.3cm}

The recommended learning sequence is:\par

\vspace{0.3cm}

\begin{enumerate}
\item Start with the video lecture to gain an intuitive, big-picture understanding.
\item Read the notes from the corresponding chapter to clarify details, solidify concepts, and explore more formal explanations.
\item Revisit the video or specific parts of the chapter if some ideas feel unclear—the two formats reinforce each other.
\item Use diagrams and worked examples as anchors for your understanding; architecture is highly visual and spatial.
\item Progress through sections sequentially, as many topics build directly on earlier ones.
\end{enumerate}

\vspace{0.5cm}

For review, you may find it helpful to skim chapter summaries and rewatch short segments of the videos rather than re-reading entire chapters.\par

\clearpage

% Update section command to format lecture titles in blue
\let\oldsection\section
\renewcommand{\section}[1]{{\color{accentblue}\oldsection{#1}}}

% ----------------- MAIN CONTENT -----------------
\chapter{Fundamentals}

% Lecture 1: Computer Abstractions and Technology
% Lecture 01: Computer Abstractions
% Lectures on Computer Architecture

\chapter{Computer Abstractions}

\section{Introduction}

% TODO: Add content from markdown

\section{The Big Picture of Computer Systems}

% TODO: Add content from markdown

\section{2. Instruction Set Architecture (ISA) - The Key Interface}

% TODO: Add content from markdown

\section{3. From Problem to Execution - The Translation Chain}

% TODO: Add content from markdown

\section{4. Writing Programs at Different Levels}

% TODO: Add content from markdown

\section{5. Microarchitecture Details}

% TODO: Add content from markdown

\section{6. Abstraction Concept}

% TODO: Add content from markdown

\section{7. Course Overview}

% TODO: Add content from markdown

\section{8. Performance Theme}

% TODO: Add content from markdown

\section{Key Takeaways}

% TODO: Add content from markdown

\section{Summary}

% TODO: Add content from markdown

% Note: Convert markdown content manually for best results
% Remember to:
% - Replace markdown images with \includegraphics
% - Convert code blocks to \begin{lstlisting}
% - Convert lists to \begin{itemize} or \begin{enumerate}
% - Convert bold/italic with \textbf{} and \textit{}


% % Lecture 2: Technology Trends
% Lecture 02: Technology Trends
% Lectures on Computer Architecture

\chapter{Technology Trends}

\section{Introduction}

% TODO: Add content from markdown

\section{1. Moore's Law - Foundation of Computer Technology Evolution}

% TODO: Add content from markdown

\section{2. Technology Scaling - Historical Data}

% TODO: Add content from markdown

\section{3. Feature Size Scaling - Lithography Improvements}

% TODO: Add content from markdown

\section{4. Technology Roadmaps - ITRS Predictions}

% TODO: Add content from markdown

\section{5. Why Smaller Transistors Improve Performance}

% TODO: Add content from markdown

\section{6. Clock Rate Trends - The Power Wall}

% TODO: Add content from markdown

\section{7. Shift to Multi-Core Processors}

% TODO: Add content from markdown

\section{8. Computer System Organization - Three Layers}

% TODO: Add content from markdown

\section{9. From High-Level Code to Machine Code - The Translation Process}

% TODO: Add content from markdown

\section{10. Program Execution - Inside the CPU}

% TODO: Add content from markdown

\section{11. Real CPU Layout - AMD Barcelona Example}

% TODO: Add content from markdown

\section{Key Takeaways}

% TODO: Add content from markdown

\section{Summary}

% TODO: Add content from markdown

% Note: Convert markdown content manually for best results
% Remember to:
% - Replace markdown images with \includegraphics
% - Convert code blocks to \begin{lstlisting}
% - Convert lists to \begin{itemize} or \begin{enumerate}
% - Convert bold/italic with \textbf{} and \textit{}


% % Lecture 3: Understanding Performance
% % Lecture 03: Understanding Performance
% Lectures on Computer Architecture

\chapter{Understanding Performance}

\section{Introduction}

% TODO: Add content from markdown

\section{1. Defining and Measuring Performance}

% TODO: Add content from markdown

\section{2. CPU Time and Performance Factors}

% TODO: Add content from markdown

\section{3. Understanding CPI in Detail}

% TODO: Add content from markdown

\section{4. Performance Optimization Principles}

% TODO: Add content from markdown

\section{5. Complete Performance Analysis}

% TODO: Add content from markdown

\section{6. Practical Performance Considerations}

% TODO: Add content from markdown

\section{Key Takeaways}

% TODO: Add content from markdown

\section{Summary}

% TODO: Add content from markdown

% Note: Convert markdown content manually for best results
% Remember to:
% - Replace markdown images with \includegraphics
% - Convert code blocks to \begin{lstlisting}
% - Convert lists to \begin{itemize} or \begin{enumerate}
% - Convert bold/italic with \textbf{} and \textit{}


% \chapter{ARM Assembly Programming}

% % Lecture 4: Introduction to ARM Assembly
% \section{Lecture 4: Introduction to ARM Assembly}

\emph{By Dr. Kisaru Liyanage}

\subsection{Introduction}

This lecture introduces ARM assembly language programming, providing the foundation for understanding how high-level programs translate to machine code. We explore the ARM instruction set architecture (ISA), focusing on its RISC design philosophy, register organization, basic instruction formats, and the toolchain used for development. Understanding assembly language is essential for comprehending how processors execute programs and for optimizing performance-critical code.

\subsection{ARM Architecture Overview}

\subsubsection{RISC Philosophy}

\textbf{Reduced Instruction Set Computer (RISC)}

\begin{itemize}
\item Simple, uniform instruction format
\item Fixed instruction length (32 bits in ARM)
\item Load/store architecture (only LOAD/STORE access memory)
\item Large number of general-purpose registers
\item Few addressing modes
\item Hardware simplicity for higher clock rates

\textbf{Contrasted with CISC (Complex Instruction Set Computer)}

| Feature | RISC | CISC |
|---------|------|------|
| \textbf{Instruction Format} | Simple, uniform format | Variable-length instructions |
| \textbf{Instruction Complexity} | Simple instructions, more instructions per program | Complex operations |
| \textbf{Memory Access} | Load/store architecture (only LOAD/STORE access memory) | Memory operands in arithmetic operations |
| \textbf{Registers} | Large number of general-purpose registers | Fewer registers |
| \textbf{Hardware Design} | Hardware simplicity for higher clock rates | More complex hardware |
| \textbf{Pipelining} | Regular structure enables efficient pipelining | More difficult to pipeline |

\textbf{ARM Design Principles}

\begin{itemize}
\item Simplicity enables high performance
\item Regular instruction encoding aids decoding
\item Load/store architecture simplifies memory access
\item Large register file reduces memory traffic
\item Consistent design across instruction types

\subsubsection{ARM Registers}

\textbf{General-Purpose Registers}

\begin{itemize}
\item \textbf{R0 to R15}: 16 registers total
\item \textbf{32 bits wide}: Can hold integers, addresses, or data
\item \textbf{R0-R12}: General computation and data storage
\item \textbf{R13 (SP)}: Stack Pointer - points to top of stack
\item \textbf{R14 (LR)}: Link Register - stores return address
\item \textbf{R15 (PC)}: Program Counter - address of next instruction

\textbf{Register Usage Conventions}
\begin{figure}[h]
\centering
\includegraphics[width=0.7\textwidth]{img/Chapter%202%20ARM%20Conventions.jpg}
\caption{Computer System Abstraction Layers}
\end{figure}

R0-R3:   Argument/result registers
\begin{itemize}
\item Pass parameters to functions
\item Return values from functions
\item Scratch registers (not preserved)

R4-R11:  Local variable registers
\begin{itemize}
\item Must be preserved across function calls
\item Callee saves/restores if used

R12:     Intra-procedure-call scratch register
\begin{itemize}
\item Can be corrupted by function calls
\item Not preserved

R13 (SP): Stack Pointer
\begin{itemize}
\item Points to top of stack
\item Must always be valid

R14 (LR): Link Register
\begin{itemize}
\item Stores return address on function call
\item Contains address to return to

R15 (PC): Program Counter
\begin{itemize}
\item Always points to next instruction
\item Modifying PC changes execution flow

\textbf{Why So Many Registers?}

\begin{itemize}
\item Reduces memory accesses (faster than cache/RAM)
\item Enables register allocation by compiler
\item Supports efficient function calls
\item Improves performance through locality

\subsubsection{Memory Organization}

\textbf{Little-Endian Byte Ordering}

\begin{itemize}
\item Least significant byte at lowest address
\item Example: 0x12345678 stored as:

\begin{verbatim}
Address:  [base+0] [base+1] [base+2] [base+3]
Content:     78       56       34       12
\end{verbatim}

\textbf{Word Alignment}

\begin{itemize}
\item Words are 32 bits (4 bytes)
\item Word addresses should be multiples of 4
\item Accessing unaligned words may cause errors or slowdown

\textbf{Address Space}

\begin{itemize}
\item 32-bit addresses can access 2³² bytes = 4 GB
\item Byte-addressable memory
\item Instructions and data in same address space (Von Neumann architecture)

\subsection{ARM Instruction Format}

\subsubsection{Instruction Structure}

\textbf{Fixed 32-Bit Length}

\begin{itemize}
\item Every instruction exactly 32 bits
\item Simplifies instruction fetch and decode
\item Enables predictable pipeline operation

\textbf{Typical Instruction Fields}

\begin{verbatim}
[Condition][Opcode][Operands]
  4 bits    varies   varies
\end{verbatim}

\textbf{Example: ADD Instruction}

\begin{lstlisting}[language=assembly]
ADD R1, R2, R3    ; R1 = R2 + R3
\end{verbatim}

Encoding includes:
\begin{itemize}
\item Condition code (usually "always")
\item Opcode for ADD operation
\item Destination register (R1)
\item Source register 1 (R2)
\item Source register 2 (R3)

\subsubsection{Instruction Types}

\textbf{Data Processing Instructions}

\begin{itemize}
\item Arithmetic: ADD, SUB, RSB (reverse subtract)
\item Logical: AND, ORR, EOR (XOR), BIC (bit clear)
\item Comparison: CMP, CMN, TST, TEQ
\item Move: MOV, MVN (move negated)
\item Shift/Rotate: LSL, LSR, ASR, ROR

\textbf{Data Transfer Instructions}

\begin{itemize}
\item Load: LDR (word), LDRB (byte), LDRH (halfword)
\item Store: STR (word), STRB (byte), STRH (halfword)
\item Multiple: LDM, STM (load/store multiple registers)

\textbf{Control Flow Instructions}

\begin{itemize}
\item Branch: B (unconditional), BEQ, BNE, BGE, BLT, etc.
\item Function call: BL (branch and link)
\item Return: MOV PC, LR

\subsubsection{Operand Types}

\textbf{Register Operands}

\begin{lstlisting}[language=assembly]
ADD R0, R1, R2    ; R0 = R1 + R2 (all registers)
\end{verbatim}

\textbf{Immediate Operands}

\begin{lstlisting}[language=assembly]
ADD R0, R1, #5    ; R0 = R1 + 5 (# indicates immediate)
MOV R2, #100      ; R2 = 100
\end{verbatim}

\textbf{Immediate Value Constraints}

\begin{itemize}
\item Limited to certain patterns due to 32-bit instruction encoding
\item 8-bit immediate + 4-bit rotation
\item Assembler warns if immediate cannot be encoded

\textbf{Shifted Register Operands}

\begin{lstlisting}[language=assembly]
ADD R0, R1, R2, LSL #2    ; R0 = R1 + (R2 << 2)
SUB R3, R4, R5, LSR #1    ; R3 = R4 - (R5 >> 1)
\end{verbatim}

\subsection{Basic ARM Instructions}

\subsubsection{Arithmetic Instructions}

\textbf{Addition}

\begin{lstlisting}[language=assembly]
ADD Rd, Rn, Rm       ; Rd = Rn + Rm
ADD Rd, Rn, #imm     ; Rd = Rn + immediate
\end{verbatim}

Examples:
\begin{lstlisting}[language=assembly]
ADD R0, R1, R2       ; R0 = R1 + R2
ADD R3, R3, #1       ; R3 = R3 + 1 (increment)
\end{verbatim}

\textbf{Subtraction}

\begin{lstlisting}[language=assembly]
SUB Rd, Rn, Rm       ; Rd = Rn - Rm
SUB Rd, Rn, #imm     ; Rd = Rn - immediate
RSB Rd, Rn, #imm     ; Rd = immediate - Rn (reverse subtract)
\end{verbatim}

Examples:
\begin{lstlisting}[language=assembly]
SUB R0, R1, R2       ; R0 = R1 - R2
SUB R4, R4, #10      ; R4 = R4 - 10 (decrement)
RSB R5, R6, #0       ; R5 = 0 - R6 (negate)
\end{verbatim}

\textbf{Multiplication} (covered in later tutorials)

\begin{lstlisting}[language=assembly]
MUL Rd, Rn, Rm       ; Rd = Rn × Rm (lower 32 bits)
\end{verbatim}

\subsubsection{Logical Instructions}

\textbf{AND Operation}

\begin{lstlisting}[language=assembly]
AND Rd, Rn, Rm       ; Rd = Rn AND Rm
AND Rd, Rn, #imm     ; Rd = Rn AND immediate
\end{verbatim}

Usage: Bit masking, clearing specific bits

Example:
\begin{lstlisting}[language=assembly]
AND R0, R0, #0xFF    ; Keep only lower 8 bits
\end{verbatim}

\textbf{OR Operation}

\begin{lstlisting}[language=assembly]
ORR Rd, Rn, Rm       ; Rd = Rn OR Rm (ORR in ARM)
ORR Rd, Rn, #imm     ; Rd = Rn OR immediate
\end{verbatim}

Usage: Setting specific bits

Example:
\begin{lstlisting}[language=assembly]
ORR R1, R1, #0x80    ; Set bit 7
\end{verbatim}

\textbf{Exclusive OR}

\begin{lstlisting}[language=assembly]
EOR Rd, Rn, Rm       ; Rd = Rn XOR Rm
EOR Rd, Rn, #imm     ; Rd = Rn XOR immediate
\end{verbatim}

Usage: Toggling bits, fast comparison

Example:
\begin{lstlisting}[language=assembly]
EOR R2, R2, R2       ; R2 = 0 (XOR with itself)
\end{verbatim}

\textbf{Move and Move Not}

\begin{lstlisting}[language=assembly]
MOV Rd, Rm           ; Rd = Rm
MOV Rd, #imm         ; Rd = immediate
MVN Rd, Rm           ; Rd = NOT Rm (bitwise complement)
\end{verbatim}

Examples:
\begin{lstlisting}[language=assembly]
MOV R0, R1           ; Copy R1 to R0
MOV R2, #0           ; Clear R2
MVN R3, R4           ; R3 = ~R4 (invert all bits)
\end{verbatim}

\subsubsection{Shift Operations}

\textbf{Logical Shift Left (LSL)}

\begin{lstlisting}[language=assembly]
LSL Rd, Rn, #shift   ; Rd = Rn << shift
MOV Rd, Rn, LSL #shift
\end{verbatim}

Effect: Multiplies by 2^shift

Example:
\begin{lstlisting}[language=assembly]
LSL R0, R1, #2       ; R0 = R1 × 4
\end{verbatim}

\textbf{Logical Shift Right (LSR)}

\begin{lstlisting}[language=assembly]
LSR Rd, Rn, #shift   ; Rd = Rn >> shift (unsigned)
MOV Rd, Rn, LSR #shift
\end{verbatim}

Effect: Divides by 2^shift (unsigned)

Example:
\begin{lstlisting}[language=assembly]
LSR R0, R1, #3       ; R0 = R1 / 8
\end{verbatim}

\textbf{Arithmetic Shift Right (ASR)}

\begin{lstlisting}[language=assembly]
ASR Rd, Rn, #shift   ; Rd = Rn >> shift (signed)
\end{verbatim}

Effect: Divides by 2^shift, preserves sign

Example:
\begin{lstlisting}[language=assembly]
ASR R0, R1, #2       ; R0 = R1 / 4 (signed)
\end{verbatim}

\textbf{Rotate Right (ROR)}

\begin{lstlisting}[language=assembly]
ROR Rd, Rn, #shift   ; Rotate Rn right by shift
\end{verbatim}

Effect: Bits rotated off right end reappear at left

Example:
\begin{lstlisting}[language=assembly]
ROR R0, R1, #8       ; Rotate R1 right by 8 bits
\end{verbatim}

\subsection{Memory Access Instructions}

\subsubsection{Load Instructions}

\textbf{Load Word (LDR)}

\begin{lstlisting}[language=assembly]
LDR Rd, [Rn]         ; Rd = Memory[Rn]
LDR Rd, [Rn, #offset]; Rd = Memory[Rn + offset]
\end{verbatim}

Examples:
\begin{lstlisting}[language=assembly]
LDR R0, [R1]         ; Load word from address in R1
LDR R2, [R3, #4]     ; Load from address R3+4
LDR R4, [R5, #-8]    ; Load from address R5-8
\end{verbatim}

\textbf{Load Byte (LDRB)}

\begin{lstlisting}[language=assembly]
LDRB Rd, [Rn, #offset]; Load one byte, zero-extend to 32 bits
\end{verbatim}

Example:
\begin{lstlisting}[language=assembly]
LDRB R0, [R1]        ; R0 = (byte at R1), upper 24 bits = 0
\end{verbatim}

\textbf{Load Halfword (LDRH)}

\begin{lstlisting}[language=assembly]
LDRH Rd, [Rn, #offset]; Load 16 bits, zero-extend to 32 bits
\end{verbatim}

Example:
\begin{lstlisting}[language=assembly]
LDRH R0, [R1, #2]    ; R0 = (halfword at R1+2), upper 16 bits = 0
\end{verbatim}

\textbf{Pseudo-Instruction for Loading Addresses}

\begin{lstlisting}[language=assembly]
LDR Rd, =label       ; Load address of label into Rd
LDR Rd, =value       ; Load 32-bit constant into Rd
\end{verbatim}

Examples:
\begin{lstlisting}[language=assembly]
LDR R0, =array       ; R0 = address of array
LDR R1, =0x12345678  ; R1 = 0x12345678 (large immediate)
\end{verbatim}

\subsubsection{Store Instructions}

\textbf{Store Word (STR)}

\begin{lstlisting}[language=assembly]
STR Rd, [Rn]         ; Memory[Rn] = Rd
STR Rd, [Rn, #offset]; Memory[Rn + offset] = Rd
\end{verbatim}

Examples:
\begin{lstlisting}[language=assembly]
STR R0, [R1]         ; Store R0 to address in R1
STR R2, [R3, #8]     ; Store R2 to address R3+8
\end{verbatim}

\textbf{Store Byte (STRB)}

\begin{lstlisting}[language=assembly]
STRB Rd, [Rn, #offset]; Store lower 8 bits of Rd
\end{verbatim}

Example:
\begin{lstlisting}[language=assembly]
STRB R0, [R1]        ; Store lower byte of R0 to address R1
\end{verbatim}

\textbf{Store Halfword (STRH)}

\begin{lstlisting}[language=assembly]
STRH Rd, [Rn, #offset]; Store lower 16 bits of Rd
\end{verbatim}

Example:
\begin{lstlisting}[language=assembly]
STRH R0, [R1, #4]    ; Store lower halfword of R0 to R1+4
\end{verbatim}

\subsubsection{Addressing Modes}

\textbf{Offset Addressing}

\begin{lstlisting}[language=assembly]
LDR R0, [R1, #4]     ; R0 = Memory[R1 + 4], R1 unchanged
\end{verbatim}

\textbf{Pre-indexed Addressing}

\begin{lstlisting}[language=assembly]
LDR R0, [R1, #4]!    ; R1 = R1 + 4, then R0 = Memory[R1]
                      ; ! indicates update base register
\end{verbatim}

\textbf{Post-indexed Addressing}

\begin{lstlisting}[language=assembly]
LDR R0, [R1], #4     ; R0 = Memory[R1], then R1 = R1 + 4
\end{verbatim}

\textbf{Register Offset}

\begin{lstlisting}[language=assembly]
LDR R0, [R1, R2]     ; R0 = Memory[R1 + R2]
LDR R0, [R1, R2, LSL #2] ; R0 = Memory[R1 + (R2 << 2)]
\end{verbatim}

\subsection{Assembly Program Structure}

\subsubsection{Directives}

\textbf{Section Directives}

\begin{lstlisting}[language=assembly]
.text                ; Code section (instructions)
.data                ; Data section (initialized variables)
.bss                 ; Uninitialized data section
\end{verbatim}

\textbf{Global and External}

\begin{lstlisting}[language=assembly]
.global main         ; Make symbol visible to linker
.extern printf       ; Declare external symbol
\end{verbatim}

\textbf{Data Definition}

\begin{lstlisting}[language=assembly]
.word value          ; Define 32-bit word
.byte value          ; Define byte
.asciz "string"      ; Define null-terminated string
.space n             ; Reserve n bytes of space
\end{verbatim}

\subsubsection{Labels}

\textbf{Purpose}

\begin{itemize}
\item Mark locations in code or data
\item Provide symbolic names for addresses
\item Enable jumps and references

\textbf{Syntax}

\begin{lstlisting}[language=assembly]
label:               ; Label for instruction
    MOV R0, #1
    ADD R1, R0, R2

array:               ; Label for data
    .word 1, 2, 3, 4
\end{verbatim}

\subsubsection{Simple Program Example}

\begin{lstlisting}[language=assembly]
    .text
    .global main

main:
    MOV R0, #5       ; R0 = 5
    MOV R1, #10      ; R1 = 10
    ADD R2, R0, R1   ; R2 = R0 + R1 = 15
    MOV R0, R2       ; R0 = R2 (return value)
    MOV PC, LR       ; Return from main

    .data
message:
    .asciz "Hello, ARM!"
\end{verbatim}

\subsection{ARM Development Tools}

\subsubsection{Toolchain Components}

\textbf{Cross-Compiler}

\begin{itemize}
\item \texttt{arm-linux-gnueabi-gcc}: Compiles C to ARM code
\item Runs on x86 PC, produces ARM binaries
\item Necessary because development machine $\neq$ target machine

\textbf{Assembler}

\begin{itemize}
\item \texttt{arm-linux-gnueabi-as}: Assembles ARM assembly to object code
\item Part of binutils package

\textbf{Linker}

\begin{itemize}
\item \texttt{arm-linux-gnueabi-ld}: Links object files to executable
\item Resolves symbols, combines code sections

\textbf{Emulator}

\begin{itemize}
\item \texttt{qemu-arm}: Emulates ARM processor on x86
\item Allows running ARM binaries on PC
\item Useful for testing without ARM hardware

\subsubsection{Compilation Process}

\textbf{From C to Executable}

\begin{verbatim}
C Source (.c)
    $\downarrow$ [gcc -S]
Assembly (.s)
    $\downarrow$ [as]
Object Code (.o)
    $\downarrow$ [ld]
Executable (a.out)
    $\downarrow$ [qemu-arm]
Execution
\end{verbatim}

\textbf{Command Examples}

\begin{lstlisting}[language=bash]
# Compile C to assembly
arm-linux-gnueabi-gcc -S program.c -o program.s

# Assemble to object code
arm-linux-gnueabi-as program.s -o program.o

# Link to executable
arm-linux-gnueabi-gcc program.o -o program

# Run with emulator
qemu-arm program
\end{verbatim}

\textbf{One-Step Compilation}

\begin{lstlisting}[language=bash]
# Compile, assemble, and link in one command
arm-linux-gnueabi-gcc program.c -o program
\end{verbatim}

\subsubsection{Debugging and Inspection}

\textbf{GDB (GNU Debugger)}

\begin{lstlisting}[language=bash]
# Debug with QEMU and GDB
qemu-arm -g 1234 program &     # Start QEMU, wait for debugger
arm-linux-gnueabi-gdb program  # Start GDB
(gdb) target remote :1234      # Connect to QEMU
(gdb) break main               # Set breakpoint
(gdb) continue                 # Run to breakpoint
(gdb) step                     # Execute one instruction
(gdb) info registers           # Show register values
\end{verbatim}

\textbf{Objdump}

\begin{lstlisting}[language=bash]
# Disassemble binary to assembly
arm-linux-gnueabi-objdump -d program
\end{verbatim}

\textbf{nm}

\begin{lstlisting}[language=bash]
# List symbols in object file
arm-linux-gnueabi-nm program.o
\end{verbatim}

\subsection{Programming in ARM Assembly}

\subsubsection{Translating C to ARM}

\textbf{C Code:}

\begin{lstlisting}[language=c]
int a = 5;
int b = 10;
int c = a + b;
\end{verbatim}

\textbf{ARM Assembly:}

\begin{lstlisting}[language=assembly]
    MOV R0, #5       ; a = 5
    MOV R1, #10      ; b = 10
    ADD R2, R0, R1   ; c = a + b
\end{verbatim}

\textbf{C Code with Array:}

\begin{lstlisting}[language=c]
int arr[3] = {1, 2, 3};
int x = arr[1];
\end{verbatim}

\textbf{ARM Assembly:}

\begin{lstlisting}[language=assembly]
    .data
arr:
    .word 1, 2, 3

    .text
    LDR R0, =arr     ; R0 = address of arr
    LDR R1, [R0, #4] ; R1 = arr[1] (offset 4 bytes)
\end{verbatim}

\subsubsection{Common Patterns}

\textbf{Clearing a Register}

\begin{lstlisting}[language=assembly]
MOV R0, #0           ; Method 1
EOR R0, R0, R0       ; Method 2 (XOR with itself)
\end{verbatim}

\textbf{Negating a Value}

\begin{lstlisting}[language=assembly]
RSB R0, R0, #0       ; R0 = 0 - R0
MVN R0, R0           ; R0 = ~R0 (bitwise, not arithmetic)
ADD R0, R0, #1       ; Then add 1 (two's complement)
\end{verbatim}

\textbf{Multiplying by Powers of 2}

\begin{lstlisting}[language=assembly]
LSL R0, R1, #3       ; R0 = R1 × 8 (faster than MUL)
\end{verbatim}

\textbf{Dividing by Powers of 2}

\begin{lstlisting}[language=assembly]
LSR R0, R1, #2       ; R0 = R1 / 4 (unsigned)
ASR R0, R1, #2       ; R0 = R1 / 4 (signed)
\end{verbatim}

\textbf{Swapping Two Registers}

\begin{lstlisting}[language=assembly]
EOR R0, R0, R1       ; XOR-based swap (no temporary)
EOR R1, R0, R1
EOR R0, R0, R1
\end{verbatim}

\subsection{Key Takeaways}

\begin{enumerate}
\item \textbf{ARM follows RISC principles} - simple instructions, load/store architecture, large register file, fixed instruction length.

\begin{enumerate}
\item \textbf{16 registers (R0-R15)} with special purposes: R13 (SP), R14 (LR), R15 (PC), and calling conventions for R0-R11.

\begin{enumerate}
\item \textbf{Three main instruction categories} - data processing (arithmetic/logic), data transfer (load/store), control flow (branches).

\begin{enumerate}
\item \textbf{Fixed 32-bit instruction format} simplifies hardware and enables efficient pipelining.

\begin{enumerate}
\item \textbf{Little-endian byte ordering} - least significant byte stored at lowest address.

\begin{enumerate}
\item \textbf{Immediate values} indicated by # symbol, with encoding constraints due to fixed instruction size.

\begin{enumerate}
\item \textbf{Memory access only through LOAD/STORE} - arithmetic operations work on registers only (load/store architecture).

\begin{enumerate}
\item \textbf{Rich addressing modes} - offset, pre-indexed, post-indexed, register offset with optional shifts.

\begin{enumerate}
\item \textbf{Cross-compilation toolchain} - arm-linux-gnueabi-gcc, as, ld, and qemu-arm for development on x86.

10. \textbf{Assembly programming requires understanding} of register allocation, instruction selection, and calling conventions.

\subsection{Summary}

ARM assembly language provides the low-level interface between software and hardware, revealing how high-level constructs translate to machine operations. The ARM architecture's RISC design emphasizes simplicity and regularity, with a uniform 32-bit instruction format, a generous 16-register set, and a clean separation between computation (using registers) and memory access (through explicit load/store instructions). Understanding ARM assembly is crucial for optimizing performance-critical code, implementing system-level software, and comprehending how processors execute programs. The development toolchain—including cross-compilers, assemblers, linkers, and emulators—enables efficient development and testing of ARM software. Mastering these fundamentals prepares us for more advanced topics including function calling conventions, stack management, and processor microarchitecture implementation.


% % Lecture 5: Number Representation and Data Processing
% % Lecture 05: Number Representation and Data Processing
% Lectures on Computer Architecture

\chapter{Number Representation and Data Processing}

\section{Introduction}

% TODO: Add content from markdown

\section{1. Number Representation Systems}

% TODO: Add content from markdown

\section{2. ARM Instruction Encoding}

% TODO: Add content from markdown

\section{3. Logical Operations}

% TODO: Add content from markdown

\section{4. Practical Bit Manipulation Examples}

% TODO: Add content from markdown

\section{Key Takeaways}

% TODO: Add content from markdown

\section{Summary}

% TODO: Add content from markdown

% Note: Convert markdown content manually for best results
% Remember to:
% - Replace markdown images with \includegraphics
% - Convert code blocks to \begin{lstlisting}
% - Convert lists to \begin{itemize} or \begin{enumerate}
% - Convert bold/italic with \textbf{} and \textit{}


% % Lecture 6: Branching
% % Lecture 06: Branching
% Lectures on Computer Architecture

\chapter{Branching}

\section{Introduction}

% TODO: Add content from markdown

\section{1. Fundamentals of Conditional Execution}

% TODO: Add content from markdown

\section{2. Comparison Instructions}

% TODO: Add content from markdown

\section{3. Conditional Branch Instructions}

% TODO: Add content from markdown

\section{4. Labels in Assembly}

% TODO: Add content from markdown

\section{5. Implementing Control Structures}

% TODO: Add content from markdown

\section{6. Array Access in Loops}

% TODO: Add content from markdown

\section{7. PC-Relative Addressing}

% TODO: Add content from markdown

\section{8. Conditional Execution (Alternative to Branching)}

% TODO: Add content from markdown

\section{9. Basic Blocks}

% TODO: Add content from markdown

\section{Key Takeaways}

% TODO: Add content from markdown

\section{Summary}

% TODO: Add content from markdown

% Note: Convert markdown content manually for best results
% Remember to:
% - Replace markdown images with \includegraphics
% - Convert code blocks to \begin{lstlisting}
% - Convert lists to \begin{itemize} or \begin{enumerate}
% - Convert bold/italic with \textbf{} and \textit{}


% % Lecture 7: Function Call and Return
% % Lecture 07: Function Call and Return
% Lectures on Computer Architecture

\chapter{Function Call and Return}

\section{Introduction}

% TODO: Add content from markdown

\section{1. Function Calling Fundamentals}

% TODO: Add content from markdown

\section{2. ARM Register Conventions}

% TODO: Add content from markdown

\section{3. Function Call Instructions}

% TODO: Add content from markdown

\section{4. Parameter Passing}

% TODO: Add content from markdown

\section{5. Return Values}

% TODO: Add content from markdown

\section{6. The Stack}

% TODO: Add content from markdown

\section{7. Stack Operations}

% TODO: Add content from markdown

\section{8. Register Preservation}

% TODO: Add content from markdown

\section{9. Nested Function Calls (Non-Leaf Functions)}

% TODO: Add content from markdown

\section{10. Recursion Example: Factorial}

% TODO: Add content from markdown

\section{11. Memory Layout and Stack vs. Heap}

% TODO: Add content from markdown

\section{Key Takeaways}

% TODO: Add content from markdown

\section{Summary}

% TODO: Add content from markdown

% Note: Convert markdown content manually for best results
% Remember to:
% - Replace markdown images with \includegraphics
% - Convert code blocks to \begin{lstlisting}
% - Convert lists to \begin{itemize} or \begin{enumerate}
% - Convert bold/italic with \textbf{} and \textit{}


% % Lecture 8: Memory Access
% % Lecture 08: Memory Access
% Lectures on Computer Architecture

\chapter{Memory Access}

\section{Introduction}

% TODO: Add content from markdown

\section{1. Character Data and Encoding}

% TODO: Add content from markdown

\section{2. Byte Load/Store Operations}

% TODO: Add content from markdown

\section{3. Half-Word Load/Store Operations}

% TODO: Add content from markdown

\section{4. String Copy Example (strcpy)}

% TODO: Add content from markdown

\section{5. Library Functions: scanf and printf}

% TODO: Add content from markdown

\section{6. Compilation, Linking, and Loading}

% TODO: Add content from markdown

\section{7. Lab Exercise Topics}

% TODO: Add content from markdown

\section{Key Takeaways}

% TODO: Add content from markdown

\section{Summary}

% TODO: Add content from markdown

% Note: Convert markdown content manually for best results
% Remember to:
% - Replace markdown images with \includegraphics
% - Convert code blocks to \begin{lstlisting}
% - Convert lists to \begin{itemize} or \begin{enumerate}
% - Convert bold/italic with \textbf{} and \textit{}


% \chapter{Processor Architecture}

% % Lecture 9: Microarchitecture and Datapath
% \input{lecture-09}

% % Lecture 10: Processor Control
% \input{lecture-10}

% % Lecture 11: Single-Cycle Execution
% \input{lecture-11}

% % Lecture 12: Pipelined Processors
% \section{Lecture 12: Pipelining and Hazards in MIPS Processors}

\emph{By Dr. Isuru Nawinne}

\subsection{Introduction}

This lecture introduces pipelining as the primary performance enhancement technique in modern processor design, transforming the inefficient single-cycle architecture into a high-throughput execution engine. We explore how pipelining applies assembly-line principles to instruction execution, dramatically improving processor throughput while maintaining individual instruction latency. The lecture examines the three fundamental types of hazards—structural, data, and control—that threaten pipeline efficiency, and discusses practical solutions including forwarding, stalling, and branch prediction that enable real-world pipelined processors to achieve near-ideal performance.

\subsection{Recap: Single-Cycle Performance Limitations}

\subsubsection{Critical Path Problem}

\textbf{Load Word as Bottleneck:}

\begin{itemize}
\item Uses most resources: Instruction Memory $\rightarrow$ Register File $\rightarrow$ ALU $\rightarrow$ Data Memory $\rightarrow$ Register File
\item Determines clock period for entire CPU
\item Forces all other instructions to wait

\textbf{Performance Issue:}

\begin{itemize}
\item Most instructions (arithmetic, branch) take less time than load
\item Jump instruction takes even less time
\item Clock period set by slowest instruction (load word)

\textbf{Design Principle Violated:}

\begin{itemize}
\item "Make the common case fast"
\item Common case (arithmetic) forced to run slowly
\item Majority of instructions underutilize available time

\subsubsection{Multi-Cycle as First Improvement}

\textbf{Basic Concept:}

\begin{itemize}
\item Divide datapath into stages
\item Each stage completes in one clock cycle
\item Shorter clock cycles than single-cycle

\textbf{Five Stages Identified:}

\begin{enumerate}
\item Instruction Fetch (IF)
\item Register Reading
\item ALU Operations
\item Memory Access
\item Register Writing

\textbf{Variable Stage Usage:}

\begin{itemize}
\item Load: Uses all 5 stages
\item Most instructions: Skip memory access (4 stages)
\item Jump: Only 2 stages (manipulating PC)

\textbf{Clock Period Determination:}

\begin{itemize}
\item Decided by slowest stage (not slowest instruction)
\item Adjust work in each stage for balance
\item Maximize utilization of each clock cycle

\textbf{Limitation:}

\begin{itemize}
\item Instruction must finish before next instruction starts
\item Hardware still idle during many cycles
\item Room for further improvement

\subsection{Pipelining Concept: The Laundry Shop Analogy}

\subsubsection{Non-Pipelined Laundry Shop}

\textbf{Setup:}

\begin{itemize}
\item One employee
\item Four customers: A, B, C, D
\item First-come, first-serve basis
\item Four stages of work per customer:
\end{itemize}
\begin{enumerate}
\item Washing: 30 minutes
\item Drying: 30 minutes
\item Folding/Ironing: 30 minutes
\item Packaging: 30 minutes
\item Total per customer: 2 hours

\textbf{Sequential Processing:}

\begin{figure}[h]
\centering
\includegraphics[width=0.7\textwidth]{img/Non-Pipelined.jpg}
\caption{Computer System Abstraction Layers}
\end{figure}

| Metric            | Value                |
| ----------------- | -------------------- |
| Total Time        | 8 hours (6pm to 2am) |
| Time per Customer | 2 hours              |
| Shop Closes       | 2am                  |

\textbf{Problems:}

\begin{itemize}
\item Machines idle while employee works on other stages
\item Washer idle during drying, folding, packaging
\item Dryer idle except during drying stage
\item Tremendous resource underutilization

\subsubsection{Pipelined Laundry Shop}

\textbf{Key Idea:}

\begin{itemize}
\item Use idle machines for next customers
\item Overlap execution of different loads
\item Parallel processing maximizes hardware utilization

\textbf{Pipelined Schedule:}

\begin{figure}[h]
\centering
\includegraphics[width=0.7\textwidth]{img/Pipelined.jpg}
\caption{Computer System Abstraction Layers}
\end{figure}

\textbf{Timeline Analysis:}

\begin{itemize}
\item 6:00-6:30: A washing (1 station busy)
\item 6:30-7:00: A drying, B washing (2 stations busy)
\item 7:00-7:30: A folding, B drying, C washing (3 stations busy)
\item 7:30-8:00: A packing, B folding, C drying, D washing (4 stations - \textbf{ALL BUSY!})
\item 8:00-8:30: A done, B packing, C folding, D drying, E washing

\textbf{Steady State:}

\begin{itemize}
\item Reached at 7:30-8:00 when all 4 stations occupied
\item Pipeline full
\item Maximum hardware utilization
\item One customer finishes every 30 minutes

\subsubsection{Performance Analysis}

\textbf{Time Comparison:}

\begin{itemize}
\item Non-pipelined: 8 hours for 4 customers
\item Pipelined: 3.5 hours for 4 customers

\textbf{Speedup Calculation:}

Speedup = Non-pipelined Time / Pipelined Time
        = 8 hours / 3.5 hours
        = 2.3$\times$

\textbf{Includes Pipeline Fill Time:}

\begin{itemize}
\item First 1.5 hours: Filling pipeline (not all stations busy)
\item After 1.5 hours: Steady state (all stations busy)

\textbf{Steady State Analysis (ignoring fill time):}

Non-pipelined: 2n hours for n loads (2 hours per load)
Pipelined: 0.5n hours for n loads (0.5 hours per load)

Steady State Speedup = 2n / 0.5n = 4$\times$

\textbf{Theoretical Maximum Speedup:}

\begin{itemize}
\item Equals number of stages
\item 4 stages $\rightarrow$ 4$\times$ speedup maximum
\item 8 stages $\rightarrow$ 8$\times$ speedup maximum (if achievable)

\subsubsection{Key Performance Terms}

\textbf{Latency:}

\begin{itemize}
\item Time to complete one individual job
\item Customer A: Still 2 hours in both cases
\item Per-instruction time unchanged

\textbf{Throughput:}

\begin{itemize}
\item How often one job completes
\item Non-pipelined: 1 job every 2 hours
\item Pipelined: 1 job every 30 minutes (steady state)
\item Throughput is the relevant metric for pipelines

\textbf{Observation:}

\begin{itemize}
\item Pipelining doesn't reduce individual job latency
\item Pipelining dramatically improves throughput
\item Overall system performance greatly enhanced

\textbf{Analogy Summary:}

\begin{itemize}
\item Customers = Instructions
\item Stages = Pipeline stages
\item Time saved = Performance improvement
\item Overlapping execution = Instruction-level parallelism

\subsection{MIPS Five-Stage Pipeline}

\subsubsection{Pipeline Stage Definitions}

\paragraph{Stage 1: Instruction Fetch (IF)}

\begin{itemize}
\item Use current Program Counter (PC)
\item PC points to next instruction to execute
\item Access instruction memory
\item Fetch instruction word
\item Duration: One clock cycle

\paragraph{Stage 2: Instruction Decode / Register Read (ID)}

\begin{itemize}
\item Decode opcode field
\item Determine instruction category
\item Identify remaining bit organization
\item Extract register addresses
\item Read register file
\item Both operations in one stage (workload balancing)

\paragraph{Stage 3: Execution (EX)}

\begin{itemize}
\item Arithmetic/Logic instructions: ALU computes result
\item Memory instructions: ALU computes address (base + offset)
\item Branch instructions: ALU performs comparison
\item One clock cycle

\paragraph{Stage 4: Memory Access (MEM)}

\begin{itemize}
\item Load instructions: Read from data memory
\item Store instructions: Write to data memory
\item Other instructions: Skip this stage
\item One clock cycle

\paragraph{Stage 5: Write Back (WB)}

\begin{itemize}
\item Write result to register file
\item Source: ALU result (arithmetic) OR memory data (load)
\item Multiplexer selects appropriate source
\item One clock cycle

\textbf{Workload Distribution Goal:}

\begin{itemize}
\item Evenly distribute work across stages
\item Minimize clock cycle time
\item Maximize hardware utilization

\subsubsection{Stage Timing Example}

\textbf{Assumed Component Delays:}

| Component           | Delay (picoseconds) |
| ------------------- | ------------------- |
| Instruction Fetch   | 200 ps              |
| Register Read/Write | 100 ps              |
| ALU Operation       | 200 ps              |
| Data Memory Access  | 200 ps              |
| Sign Extension      | negligible          |
| Multiplexers        | negligible          |

\textbf{Single-Cycle Instruction Times:}

| Instruction Type   | Stages Used     | Total Time |
| ------------------ | --------------- | ---------- |
| Load Word (LW)     | IF+ID+EX+MEM+WB | 800 ps     |
| Store Word (SW)    | IF+ID+EX+MEM    | 700 ps     |
| R-type (ADD, etc.) | IF+ID+EX+WB     | 600 ps     |
| Branch (BEQ)       | IF+ID+EX        | 500 ps     |

\textbf{Load Word Critical Path:} 800 ps determines clock period

\subsubsection{Pipeline Implementation Details}

\textbf{Clock Cycle Determination:}

\begin{itemize}
\item Must accommodate longest stage
\item Longest stage: 200 ps (IF, ALU, MEM)
\item Clock cycle: 200 ps
\item Some stages underutilize cycle (register read/write: 100 ps)

\textbf{Register Read/Write Timing:}

\begin{itemize}
\item \textbf{CRITICAL:} Register write first half, read second half of same clock cycle
\item Enables same-cycle read-after-write
\item Prevents data hazards in some cases

\textbf{Stage Alignment to Clock Cycles:}

| Stage | Work                    | Time   | Cycle Time               |
| ----- | ----------------------- | ------ | ------------------------ |
| IF    | Instruction Memory read | 200 ps | 200 ps ✓                 |
| ID    | Decode + Register Read  | 100 ps | 200 ps (space left)      |
| EX    | ALU operation           | 200 ps | 200 ps ✓                 |
| MEM   | Data Memory access      | 200 ps | 200 ps ✓                 |
| WB    | Register write          | 100 ps | 200 ps (first half only) |

\textbf{Space in ID Stage:}

\begin{itemize}
\item Register read: 100 ps
\item Decoding: Fits in remaining 100 ps
\item Combinational logic for opcode decode
\item Total: ~200 ps utilized

\textbf{Space in WB Stage:}

\begin{itemize}
\item Register write: 100 ps (first half)
\item Second half: Available for next instruction's register read

\subsubsection{Load Word Pipeline Example}

\textbf{Instruction Stream:} All Load Word instructions

\begin{lstlisting}[language=assembly]
LW $1, 0($10)
LW $2, 4($10)
LW $3, 8($10)
LW $4, 12($10)
...
\end{verbatim}

\textbf{Pipeline Timing Diagram:}

\begin{verbatim}
Time (ps):  0-200  200-400  400-600  600-800  800-1000  1000-1200
LW $1:      IF     ID       EX       MEM      WB
LW $2:             IF       ID       EX       MEM       WB
LW $3:                      IF       ID       EX        MEM
LW $4:                               IF       ID        EX
\end{verbatim}

\textbf{Single-Cycle Comparison:}

\begin{itemize}
\item Non-pipelined: 800 ps per instruction
\item Pipelined: 200 ps per instruction (after pipeline fills)

\textbf{Throughput Improvement:}

\begin{verbatim}
Non-pipelined: 1 instruction every 800 ps
Pipelined: 1 instruction every 200 ps

Speedup = 800 / 200 = 4×
\end{verbatim}

\textbf{Absolute Time per Instruction:}

\begin{itemize}
\item Still ~800 ps (slightly more with alignment overhead)
\item Latency unchanged or slightly worse
\item Throughput dramatically improved

\subsubsection{Ideal vs Actual Speedup}

\textbf{Ideal Case (balanced stages):}

\begin{verbatim}
Time between instructions (pipelined) = Time per instruction (non-pipelined) / Number of stages

Maximum Speedup = Number of Stages
\end{verbatim}

\textbf{Actual Implementation:}

\begin{itemize}
\item Stages not perfectly balanced
\item Register operations faster than memory/ALU
\item Speedup < Number of stages
\item Example: 5 stages $\rightarrow$ 4$\times$ speedup (not 5$\times$)

\textbf{Reasons for Less Than Ideal:}

\begin{enumerate}
\item Unbalanced stage delays
\item Pipeline fill time overhead
\item Hazards (discussed later)
\item Added synchronization logic

\subsection{MIPS ISA Design for Pipelining}

\subsubsection{Fixed Instruction Length}

\textbf{MIPS Characteristic:}

\begin{itemize}
\item All instructions exactly 32 bits
\item Same as ARM (also designed for pipelining)

\textbf{Benefits for Pipelining:}

\begin{itemize}
\item Simple instruction fetch (always 32 bits)
\item Simple decode (fixed format)
\item Bus width fully utilized every time
\item No variable-width handling logic

\textbf{Alternative (Variable-Length):}

\begin{itemize}
\item Complicates fetch stage
\item Requires width detection logic
\item May need multiple fetch cycles
\item Added combinational logic delays

\subsubsection{Fewer Regular Instruction Formats}

\textbf{MIPS Formats:}

\begin{itemize}
\item Only 3-4 instruction formats (R, I, J types)
\item Small opcode field (6 bits)
\item Regular register field positions

\textbf{Benefits:}

\begin{itemize}
\item Fast decoding (small opcode $\rightarrow$ simple logic)
\item Fits decode + register read in one stage
\item Minimal combinational delay

\textbf{Register Field Consistency:}

\begin{itemize}
\item RS (bits 21-25): First source register
\item RT (bits 16-20): Second source / destination
\item RD (bits 11-15): Destination (R-type)
\item Same positions across formats

\textbf{Decoding Simplification:}

\begin{itemize}
\item Small opcode $\rightarrow$ simple decode logic
\item Regular formats $\rightarrow$ minimal mux complexity
\item Fast enough for single clock cycle

\subsubsection{Separate ALU Operation Field}

\textbf{Function Field (funct):}

\begin{itemize}
\item Bits 0-5: Specifies ALU operation for R-type
\item Separate from opcode
\item Only examined for R-type (opcode = 0)

\textbf{Design Rationale:}

\begin{itemize}
\item ALU operation determined in EX stage
\item Opcode used in ID stage
\item Temporal separation matches pipeline stages

\textbf{Benefit:}

\begin{itemize}
\item funct field processed later (EX stage)
\item Opcode processed early (ID stage)
\item Separating them simplifies each stage
\item Avoids large opcode (keeps decode simple)

\textbf{Alternative Design:}

\begin{itemize}
\item Include funct in opcode
\item Larger opcode field needed
\item More complex decode logic
\item Slower ID stage
\item Worse pipeline balance

\subsubsection{Load/Store Addressing Mode}

\textbf{MIPS Addressing:}

\begin{itemize}
\item Base register + offset
\item Address = $rs + immediate
\item Calculation: Simple addition

\textbf{Pipeline Fit:}

\begin{itemize}
\item Address calculation: EX stage (ALU)
\item Memory access: MEM stage (next cycle)
\item Clean separation into two stages

\textbf{Design Philosophy:}

\begin{itemize}
\item ISA designed with pipeline in mind
\item Not optimized for single-cycle
\item Performance through pipelining

\textbf{MIPS vs Other ISAs:}

\begin{itemize}
\item MIPS: Designed for pipelining from start
\item x86: Complex instructions, harder to pipeline
\item ARM: Similar philosophy to MIPS
\item RISC principles support pipelining

\subsection{Instruction-Level Parallelism (ILP)}

\subsubsection{Parallel Execution Concept}

\textbf{Definition:}

\begin{itemize}
\item Multiple instructions executing simultaneously
\item Each at different pipeline stage
\item Overlapping execution

\textbf{Example at Steady State:}

Time Window: 800-1000 ps

Instruction A: WB stage (writing result)
Instruction B: MEM stage (memory access)
Instruction C: EX stage (ALU operation)
Instruction D: ID stage (decode, register read)
Instruction E: IF stage (fetch)

Five instructions active simultaneously!

\textbf{Instruction-Level Parallelism (ILP):}

\begin{itemize}
\item Lowest granularity of parallelism
\item Inside CPU microarchitecture
\item Transparent to software
\item Hardware manages parallelism

\subsubsection{Levels of Parallelism}

\textbf{Instruction-Level Parallelism:}

\begin{itemize}
\item Multiple instructions in pipeline
\item Same program/thread
\item Within CPU core
\item Microsecond/nanosecond scale

\textbf{Thread-Level Parallelism:}

\begin{itemize}
\item Multiple threads on same core
\item Context switching
\item OS-managed
\item Millisecond scale

\textbf{Program-Level Parallelism:}

\begin{itemize}
\item Multiple programs/processes
\item Multi-core execution
\item OS-scheduled
\item Varied time scales

\textbf{Application-Level Parallelism:}

\begin{itemize}
\item Distributed computing
\item Multiple machines
\item Network communication
\item Seconds to minutes scale

\textbf{ILP Focus:}

\begin{itemize}
\item Fine-grained parallelism
\item Hardware implementation
\item Transparent to programmer (mostly)
\item Foundation for all higher levels

\subsection{Pipeline Hazards: Structural Hazards}

\subsubsection{Hazard Definition}

\textbf{General Concept:}

\begin{itemize}
\item Situations preventing next instruction from starting
\item Violates basic pipelining goal
\item Reduces throughput
\item Requires pipeline stalls (bubbles)

\textbf{Three Categories:}

\begin{enumerate}
\item \textbf{Structural Hazards:} Hardware resource busy
\item \textbf{Data Hazards:} Need data from previous instruction
\item \textbf{Control Hazards:} Decision depends on previous result

\subsubsection{Structural Hazard: Single Memory}

\textbf{Scenario:}

\begin{itemize}
\item Single memory device for both instructions and data
\item No separate instruction/data memory
\item Same device holds program and data

\textbf{Conflict Example:}

Time:    0-200   200-400  400-600  600-800
LW $1:   IF      ID       EX       MEM
LW $2:           IF       ID       EX
LW $3:                    IF       ID
LW $4:                             IF  $\leftarrow$ CONFLICT!

At 600-800 ps:
\begin{itemize}
\item LW $1 needs data memory (MEM stage)
\item LW $4 needs instruction memory (IF stage)
\item Same physical memory device!
\item Cannot access simultaneously

\textbf{Problem:}

\begin{itemize}
\item Memory can only service one request per cycle
\item Instruction fetch AND data access conflict
\item Hardware resource (memory) busy

\subsubsection{Pipeline Stall (Bubble)}

\textbf{Solution: Insert Bubble}

\begin{verbatim}
Time:    0-200   200-400  400-600  600-800  800-1000  1000-1200
LW $1:   IF      ID       EX       MEM      WB
LW $2:           IF       ID       EX       [BUBBLE]  MEM
LW $3:                    IF       ID       EX        [BUBBLE]
LW $4:                             IF       [BUBBLE]  ID
\end{verbatim}

\textbf{Bubble Characteristics:}

\begin{itemize}
\item No instruction in that pipeline stage
\item Like air bubble in water pipeline
\item Hardware idle for that stage
\item Wastes one clock cycle
\item Propagates through pipeline stages

\textbf{Impact:}

\begin{itemize}
\item One instruction delayed
\item Subsequent instructions delayed
\item Throughput reduced
\item Performance loss

\textbf{Bubble Analogy:}

\begin{itemize}
\item Water pipeline: Continuous flow
\item Air bubble: Break in flow
\item Takes time to propagate through
\item Reduces effective flow rate

\subsubsection{Solutions to Structural Hazards}

\textbf{Solution 1: Separate Memories}

\begin{itemize}
\item Instruction memory separate from data memory
\item Harvard architecture
\item Simultaneous access possible
\item No structural hazard

\textbf{Solution 2: Separate Caches}

\begin{itemize}
\item Single main memory
\item Separate instruction cache (I-cache)
\item Separate data cache (D-cache)
\item Cache: Fast buffer between CPU and memory
\item Caches can be accessed simultaneously
\item Details in future lectures (memory hierarchy)

\textbf{Design Recommendation:}

\begin{itemize}
\item Modern processors use separate caches
\item Necessary for high-performance pipelining
\item Small area overhead for large performance gain

\subsection{Data Hazards}

\subsubsection{Data Hazard Definition}

\textbf{Concept:}

\begin{itemize}
\item Subsequent instruction needs data from previous instruction
\item Data not yet available (still being computed/written)
\item Reading too early $\rightarrow$ wrong value
\item Writing too early $\rightarrow$ data corruption

\textbf{Example:}

\begin{lstlisting}[language=assembly]
ADD $s0, $t0, $t1      # $s0 = $t0 + $t1
SUB $t2, $s0, $t3      # $t2 = $s0 - $t3 (uses $s0 from ADD)
\end{verbatim}

\textbf{Problem:}

\begin{itemize}
\item ADD computes $s0 value in EX stage
\item SUB needs $s0 value in ID stage (register read)
\item Timing mismatch

\subsubsection{Data Hazard Example Analysis}

\textbf{Instruction Sequence:}

\begin{lstlisting}[language=assembly]
ADD $s0, $t0, $t1
SUB $t2, $s0, $t3
\end{verbatim}

\textbf{Pipeline Without Stalls:}

\begin{verbatim}
Time:    0-200   200-400  400-600  600-800  800-1000
ADD:     IF      ID       EX       MEM      WB
SUB:             IF       ID       EX       MEM
                          ↑
                    Reads $s0 here (old value!)

ADD writes $s0 here $\downarrow$
\end{verbatim}

\textbf{Problem Timeline:}

\begin{itemize}
\item 200-400: ADD reads $t0, $t1; SUB fetched
\item 400-600: ADD computes in ALU; SUB reads registers (gets OLD $s0!)
\item 600-800: ADD result available but not in register yet
\item 800-1000: ADD writes $s0 to register (first half of cycle)

SUB reads $s0 at 400-600, but correct value not available until 800-1000!

\subsubsection{Solution 1: Pipeline Stalls}

\textbf{Insert Two Bubbles:}

\begin{verbatim}
Time:    0-200   200-400  400-600  600-800  800-1000  1000-1200  1200-1400
ADD:     IF      ID       EX       MEM      WB
[BUBBLE]                  IF       [BUBBLE] [BUBBLE]
[BUBBLE]                           IF       [BUBBLE]
SUB:                                         IF        ID
\end{verbatim}

\textbf{Result:}

\begin{itemize}
\item SUB fetched at 1000-1200
\item SUB reads registers at 1200-1400 (second half at 1200)
\item ADD writes $s0 at 800-1000 (first half at 800)
\item Sufficient time gap: Correct value available

\textbf{Cost:}

\begin{itemize}
\item Two clock cycles wasted
\item Throughput reduced
\item Performance penalty

\textbf{Critical Timing:}

\begin{itemize}
\item Register write: First half of WB cycle
\item Register read: Second half of ID cycle
\item Enables back-to-back reading of just-written value

\subsubsection{Solution 2: Forwarding (Bypassing)}

\textbf{Key Observation:}

\begin{itemize}
\item ADD result available after EX stage (400-600)
\item Result at ALU output
\item Not yet written to register file
\item But SUB's ALU operation at 600-800
\item Can forward ALU output directly to ALU input!

\textbf{Forwarding Logic:}

\begin{verbatim}
Time:    0-200   200-400  400-600  600-800  800-1000
ADD:     IF      ID       EX       MEM      WB
SUB:             IF       ID       EX       MEM
                          ↑        ↑
                    Read regs   Use forwarded value!
\end{verbatim}

\textbf{Implementation:}

\begin{itemize}
\item Multiplexer at ALU input
\item Selects between:
\item Register file output (normal path)
\item Forwarded value from previous ALU output
\item Control logic detects dependency
\item Routes correct value

\textbf{Benefit:}

\begin{itemize}
\item Eliminates two stalls
\item No performance penalty
\item Requires additional hardware:
\item Forwarding multiplexers
\item Forwarding detection logic
\item Forwarding paths (wires)
\item Pipeline registers to hold values

\textbf{Complexity:}

\begin{itemize}
\item Careful synchronization required
\item Detect true dependencies
\item Avoid false positives
\item Additional control signals

\textbf{Result:}

\begin{itemize}
\item SUB can execute immediately after ADD
\item No stalls needed
\item Correct value forwarded

\subsubsection{Load-Use Data Hazard}

\textbf{Special Case:}

\begin{lstlisting}[language=assembly]
LW  $s0, 0($t0)        # Load from memory into $s0
SUB $t2, $s0, $t3      # Use $s0 immediately
\end{verbatim}

\textbf{Problem:}

\begin{itemize}
\item Load result available after MEM stage (data from memory)
\item SUB needs value in EX stage
\item Even forwarding can't help!

\textbf{Timeline:}

\begin{verbatim}
Time:    0-200   200-400  400-600  600-800  800-1000
LW:      IF      ID       EX       MEM      WB
SUB:             IF       ID       EX       MEM
                          ↑        ↑
                    Need value   Value first available here!
\end{verbatim}

LW result available at 600-800, but SUB's EX at 600-800 (simultaneous!)

\textbf{Unavoidable Stall:}

\begin{verbatim}
Time:    0-200   200-400  400-600  600-800  800-1000  1000-1200
LW:      IF      ID       EX       MEM      WB
[BUBBLE]                  IF       [BUBBLE] ID
SUB:                                         IF        ID
\end{verbatim}

\textbf{One stall bubble required:}

\begin{itemize}
\item Cannot be eliminated by forwarding
\item Can forward from MEM to EX (saves one stall vs two)
\item But at least one stall unavoidable

\subsubsection{Compiler Solution: Code Reordering}

\textbf{C Code Example:}

\begin{lstlisting}[language=c]
a = b + e;
c = b + f;
\end{verbatim}

\textbf{Naive Assembly (Load-Use Hazards):}

\begin{lstlisting}[language=assembly]
LW   $t1, 0($t0)    # Load b into $t1
LW   $t2, 4($t0)    # Load e into $t2
ADD  $t3, $t1, $t2  # a = b + e ← HAZARD: uses $t2 immediately after LW
SW   $t3, 8($t0)    # Store a

LW   $t4, 12($t0)   # Load f into $t4
ADD  $t5, $t1, $t4  # c = b + f ← HAZARD: uses $t4 immediately after LW
SW   $t5, 16($t0)   # Store c
\end{verbatim}

\textbf{Total:} 7 instructions + 2 stalls = 9 clock cycles

\textbf{Optimized Assembly (Reordered):}

\begin{lstlisting}[language=assembly]
LW   $t1, 0($t0)    # Load b into $t1
LW   $t2, 4($t0)    # Load e into $t2
LW   $t4, 12($t0)   # Load f into $t4 ← Moved here!
ADD  $t3, $t1, $t2  # a = b + e ← No hazard! $t2 available
SW   $t3, 8($t0)    # Store a ← Moved here!
ADD  $t5, $t1, $t4  # c = b + f ← No hazard! $t4 available
SW   $t5, 16($t0)   # Store c
\end{verbatim}

\textbf{Total:} 7 instructions + 0 stalls = 7 clock cycles

\textbf{Technique:}

\begin{itemize}
\item Load f earlier (between loading b and e)
\item Fills stall slot with useful work
\item Store a before second ADD (fills another gap)
\item No bubbles needed

\textbf{Savings:} 2 clock cycles (22% improvement)

\textbf{Compiler Responsibility:}

\begin{itemize}
\item Analyze dependencies
\item Reorder instructions safely
\item Fill stall slots with independent instructions
\item Maintain program semantics

\textbf{Programmer Awareness:}

\begin{itemize}
\item Understand pipeline behavior
\item Write code amenable to reordering
\item Separate dependent instructions when possible
\item Help compiler optimize

\subsection{Control Hazards}

\subsubsection{Control Hazard Definition}

\textbf{Concept:}

\begin{itemize}
\item Branch/Jump outcome determines next instruction
\item Decision depends on previous computation
\item Can't fetch next instruction until decision made
\item Pipeline must wait

\textbf{Example:}

\begin{lstlisting}[language=assembly]
BEQ $1, $2, target     # Branch if $1 == $2
ADD $3, $4, $5         # Next sequential instruction
...
target: SUB $6, $7, $8 # Branch target
\end{verbatim}

\textbf{Which instruction to fetch after BEQ?}

\begin{itemize}
\item ADD if branch NOT taken
\item SUB if branch IS taken
\item Decision requires comparison: $1 vs $2

\subsubsection{Branch Execution in Pipeline}

\textbf{Branch Instruction:}

\begin{lstlisting}[language=assembly]
BEQ $1, $2, 40         # Branch 40 instructions ahead if equal
\end{verbatim}

\textbf{Pipeline Stages:}

\begin{enumerate}
\item IF: Fetch BEQ instruction
\item ID: Read $1, $2 from register file
\item EX: ALU compares (subtract $2 from $1, check zero flag)
\item Result available after EX stage

\textbf{Problem:}

\begin{itemize}
\item Next instruction fetch at cycle 2 (IF for next instruction)
\item Branch outcome known at cycle 3 (after EX)
\item Must guess which instruction to fetch!

\textbf{Without Optimization:}

\begin{verbatim}
Time:    0-200   200-400  400-600  600-800
BEQ:     IF      ID       EX       MEM
???:             IF       ???
\end{verbatim}

Two bubbles required if wait for outcome

\subsubsection{Solution 1: Early Branch Resolution}

\textbf{Add Hardware in ID Stage:}

\begin{itemize}
\item Small adder for comparison
\item Compute branch condition early (ID instead of EX)
\item Subtract $1 - $2 in ID stage
\item Parallel to register read

\textbf{Modified Pipeline:}

\begin{verbatim}
Time:    0-200   200-400  400-600
BEQ:     IF      ID       EX
                 ↑
          Decision here!
Next:            IF
\end{verbatim}

\textbf{Benefit:}

\begin{itemize}
\item Decision after ID (one cycle earlier)
\item Only one bubble needed (vs two)
\item Better performance

\textbf{Cost:}

\begin{itemize}
\item Additional adder hardware
\item Extra combinational logic in ID stage
\item More complex ID stage

\textbf{Limitation:}

\begin{itemize}
\item Still one unavoidable stall
\item Can't know outcome in same cycle as fetch

\subsubsection{Solution 2: Branch Prediction}

\textbf{Static Branch Prediction:}

\begin{itemize}
\item Guess branch outcome
\item Fetch based on guess
\item If correct: No penalty
\item If wrong: Discard fetched instruction, fetch correct one

\textbf{Strategy: Predict Not Taken}

\begin{itemize}
\item Assume branch will NOT be taken
\item Always fetch PC + 4 (sequential instruction)
\item Proceed normally if correct
\item Stall and correct if wrong

\textbf{Example (Prediction Correct):}

\begin{lstlisting}[language=assembly]
ADD  $3, $4, $5
BEQ  $1, $2, 14        # Actually NOT taken
LW   $8, 0($9)         # Fetch this (prediction: not taken)
\end{verbatim}

\textbf{Timeline:}

\begin{verbatim}
Time:    0-200   200-400  400-600  600-800
ADD:     IF      ID       EX       MEM
BEQ:             IF      ID       EX
LW:                      IF       ID
                         ↑ Fetched based on prediction
\end{verbatim}

At 400-600 (after BEQ's ID):

\begin{itemize}
\item Determine branch NOT taken
\item Prediction correct!
\item LW continues normally
\item No stall!

\textbf{Example (Prediction Incorrect):}

\begin{lstlisting}[language=assembly]
ADD  $3, $4, $5
BEQ  $1, $2, 14        # Actually IS taken
LW   $8, 0($9)         # Fetched (but shouldn't execute)
...
target: SUB $6, $7, $8 # Should execute this instead
\end{verbatim}

\textbf{Timeline:}

\begin{verbatim}
Time:    0-200   200-400  400-600  600-800
ADD:     IF      ID       EX       MEM
BEQ:             IF      ID       EX
LW:                      IF       [DISCARD]
SUB:                              IF
\end{verbatim}

At 400-600 (after BEQ's ID):

\begin{itemize}
\item Determine branch IS taken
\item Prediction wrong!
\item Discard LW (clear pipeline stage)
\item Fetch SUB from branch target
\item One bubble inserted

\textbf{Result Analysis:}

\begin{itemize}
\item Correct prediction: Save one cycle
\item Incorrect prediction: Same as no prediction (one stall)
\item Net benefit if prediction often correct
\item No additional penalty for wrong guess

\subsubsection{Static Branch Prediction Strategies}

\textbf{Simple Static: Always Predict Not Taken}

\begin{itemize}
\item Fixed prediction
\item Ignore branch type
\item Ignore branch history
\item Simple hardware

\textbf{Program Behavior-Based Static:}

\begin{itemize}
\item Analyze typical branch patterns
\item Make predictions based on code structure

\textbf{Backward Branches:}

\begin{itemize}
\item Usually taken
\item Example: Loops

\begin{lstlisting}[language=assembly]
loop:
    ...
    BEQ $t0, $zero, loop   # Backward branch
\end{verbatim}

\begin{itemize}
\item Loop iterations: Branch taken many times
\item Loop exit: Branch not taken once
\item Prediction: Taken $\rightarrow$ Correct most of time

\textbf{Forward Branches:}

\begin{itemize}
\item Usually not taken
\item Example: If statements

\begin{lstlisting}[language=assembly]
    BEQ $t0, $zero, skip
    ...                      # True case
\end{verbatim}
skip:
    ...                      # After if

\begin{itemize}
\item True case: Branch not taken
\item False case: Branch taken
\item Prediction depends on code style

\textbf{Strategy: Backward Taken, Forward Not Taken}

\begin{itemize}
\item 90%+ accuracy possible
\item Based on empirical program analysis
\item Requires code analysis

\subsubsection{Dynamic Branch Prediction}

\textbf{Concept:}

\begin{itemize}
\item Hardware learns branch behavior
\item Predicts based on history
\item Adapts to current code execution
\item Not fixed prediction

\textbf{Branch History Table:}

\begin{itemize}
\item Hardware table storing recent branch outcomes
\item Indexed by branch instruction address
\item Each entry: Branch taken or not taken recently
\item Predicts based on recent behavior

\textbf{Simple 1-Bit Predictor:}

\begin{itemize}
\item One bit per branch: Last outcome
\item Predict same as last time
\item Updates after each execution

\textbf{Example:}

\begin{verbatim}
Loop iteration 1: Taken $\rightarrow$ Predict taken next
Loop iteration 2: Taken $\rightarrow$ Predict taken next
...
Loop iteration 100: Taken $\rightarrow$ Predict taken next
Loop exit: Not taken $\rightarrow$ Predict not taken next (wrong for next loop!)
\end{verbatim}

Problem: Wrong twice per loop (entry and exit)

\textbf{2-Bit Saturating Counter:}

\begin{itemize}
\item Two bits per branch: State machine
\item Four states:
\item 00: Strongly not taken
\item 01: Weakly not taken
\item 10: Weakly taken
\item 11: Strongly taken
\item Change prediction after two consecutive wrong predictions
\item More stable

\textbf{Advanced Predictors:}

\begin{itemize}
\item Correlating predictors (look at multiple branches)
\item Two-level adaptive predictors
\item Tournament predictors (combine multiple algorithms)
\item Very high accuracy (>95%)

\textbf{Hardware Cost:}

\begin{itemize}
\item Branch history table (memory)
\item Prediction logic (comparators, counters)
\item Update logic
\item Worthwhile for performance gain

\subsection{Summary and Key Concepts}

\subsubsection{Pipelining Benefits}

\textbf{Performance Improvement:}

\begin{itemize}
\item Throughput increased by number of stages
\item 5-stage pipeline $\rightarrow$ 4-5$\times$ speedup
\item Latency unchanged or slightly worse
\item Overlapping execution key

\textbf{Hardware Utilization:}

\begin{itemize}
\item All stages active in steady state
\item Parallel processing
\item Maximum efficiency

\subsubsection{Pipeline Challenges}

\textbf{Hazards:}

\begin{enumerate}
\item \textbf{Structural:} Hardware resource conflicts
\item \textbf{Data:} Instruction dependencies
\item \textbf{Control:} Branch/jump decisions

\textbf{Solutions:}

\begin{itemize}
\item Structural: Separate memories/caches
\item Data: Forwarding, stalls, code reordering
\item Control: Early resolution, branch prediction

\subsubsection{MIPS Design Philosophy}

\textbf{ISA Designed for Pipelining:}

\begin{itemize}
\item Fixed 32-bit instruction length
\item Regular instruction formats
\item Separate funct field
\item Simple addressing modes
\item Balanced pipeline stages

\textbf{Performance Through Hardware:}

\begin{itemize}
\item Pipelining fundamental to MIPS
\item Not optimized for single-cycle
\item Hardware complexity for software simplicity

\subsubsection{Key Takeaways}

\begin{enumerate}
\item Pipelining improves throughput, not latency
\item Steady state determines peak performance
\item Pipeline fill time overhead for small programs
\item Hazards reduce pipelining efficiency
\item Forwarding eliminates many data hazards
\item Load-use hazard always requires one stall
\item Branch prediction crucial for control flow
\item Compiler optimization reduces stalls
\item ISA design significantly impacts pipeline efficiency
\end{enumerate}

10. ILP fundamental to modern processor performance

\subsection{Important Formulas and Metrics}

\subsubsection{Speedup Calculation}

\begin{verbatim}
Speedup = Non-pipelined Time / Pipelined Time

Ideal Speedup = Number of Pipeline Stages

Actual Speedup = Number of Stages / (1 + Hazard Impact)
\end{verbatim}

\subsubsection{Throughput}

\begin{verbatim}
Throughput = 1 instruction / Clock Period

Throughput Improvement = Clock Period (non-pipelined) / Clock Period (pipelined)
\end{verbatim}

\subsubsection{Pipeline Performance}

\begin{verbatim}
Time = (Number of Instructions + Stages - 1) × Clock Period

CPI (Cycles Per Instruction) = 1 + Stall Cycles per Instruction

Effective CPI = 1 + (Structural Stalls + Data Stalls + Control Stalls)
\end{verbatim}

\subsubsection{Branch Prediction Accuracy}

\begin{verbatim}
Accuracy = Correct Predictions / Total Branches

Stall Reduction = Accuracy × Cycles Saved per Correct Prediction
\end{verbatim}

\subsection{Key Takeaways}

\begin{enumerate}
\item \textbf{Pipelining improves throughput, not latency}—individual instructions take same or longer time, but more instructions complete per unit time.

\begin{enumerate}
\item \textbf{Five-stage MIPS pipeline}: Instruction Fetch (IF), Instruction Decode (ID), Execute (EX), Memory Access (MEM), Write-Back (WB).

\begin{enumerate}
\item \textbf{Ideal speedup equals number of stages}—five-stage pipeline theoretically achieves 5$\times$ speedup over single-cycle design.

\begin{enumerate}
\item \textbf{Assembly line analogy clarifies concept}—like manufacturing, each stage works on different item simultaneously for maximum efficiency.

\begin{enumerate}
\item \textbf{Pipeline registers store intermediate results} between stages, enabling independent operation and preventing data corruption.

\begin{enumerate}
\item \textbf{Three hazard types threaten pipeline efficiency}: Structural (resource conflicts), Data (register dependencies), Control (branch/jump delays).

\begin{enumerate}
\item \textbf{Structural hazards resolved by hardware duplication}—separate instruction and data caches eliminate memory access conflicts.

\begin{enumerate}
\item \textbf{Data hazards occur when instructions depend on previous results}—forwarding (bypassing) allows ALU results to skip write-back stage.

\begin{enumerate}
\item \textbf{Forwarding paths connect pipeline stages directly}, enabling result use before register file write completes.

10. \textbf{Load-use hazard requires one-cycle stall}—memory data unavailable in time for immediate ALU use even with forwarding.

11. \textbf{Compiler code reordering can eliminate some stalls}—moving independent instructions into load delay slots maintains pipeline flow.

12. \textbf{Control hazards arise from branch/jump instructions}—don't know next PC until branch resolves in third cycle.

13. \textbf{Branch delay of 3 cycles} in basic pipeline—fetch/decode/execute complete before decision known, wasting 3 instruction slots.

14. \textbf{Early branch resolution reduces penalty}—dedicated comparison hardware in ID stage cuts delay to 1 cycle.

15. \textbf{Static branch prediction} assumes direction (e.g., always not-taken)—simple but limited effectiveness.

16. \textbf{Dynamic branch prediction} learns patterns from history—branch target buffer with 2-bit saturating counters achieves >90% accuracy.

17. \textbf{Two-bit counters prevent single misprediction disruption}—requires two wrong predictions to change direction, handling loop patterns well.

18. \textbf{Pipeline performance} = 1 CPI + Structural Stalls + Data Stalls + Control Stalls—minimizing hazards approaches ideal throughput.

19. \textbf{Modern processors use sophisticated prediction}—multi-level predictors, pattern history tables, and return address stacks minimize control hazards.

20. \textbf{Pipeline complexity trades off with performance}—deeper pipelines increase throughput but amplify hazard penalties and design difficulty.

\subsection{Summary}

Pipelining revolutionizes processor performance by applying manufacturing assembly-line principles to instruction execution, allowing multiple instructions to occupy different pipeline stages simultaneously. The five-stage MIPS pipeline (IF, ID, EX, MEM, WB) theoretically achieves 5$\times$ speedup by keeping all hardware components busy every cycle, transforming the inefficient single-cycle design where most hardware sat idle most of the time. However, three hazard types threaten this ideal performance: structural hazards from resource conflicts (solved by hardware duplication like separate instruction and data caches), data hazards from register dependencies (addressed by forwarding paths that bypass results directly between stages, though load-use cases still require one-cycle stalls), and control hazards from branches that don't resolve until the third cycle (mitigated by early branch resolution hardware, static prediction strategies, and sophisticated dynamic branch predictors using two-bit saturating counters that achieve over 90% accuracy). The effectiveness of forwarding demonstrates how careful hardware design can eliminate most data hazard stalls, while compiler optimizations like instruction reordering can fill remaining delay slots with useful work. Branch prediction evolution from simple static schemes to complex dynamic predictors with branch target buffers reflects the critical importance of minimizing control hazards in modern high-performance processors. Pipeline registers between stages serve as the crucial mechanism enabling independent stage operation, storing intermediate results and control signals while preventing data corruption across instruction overlaps. While pipelining introduces significant design complexity compared to single-cycle implementations, the dramatic performance improvements—approaching 5$\times$ speedup in practice—justify this added sophistication, making pipelining universal in modern processor architectures from embedded systems to supercomputers. Understanding these hazards and their solutions provides essential foundation for comprehending real-world processor implementations and the tradeoffs between pipeline depth, clock frequency, and hazard penalties that define contemporary computer architecture.


% % Lecture 13: Pipeline Operation and Timing
% \input{lecture-13}

% \chapter{Memory Systems}

% % Lecture 14: Memory Hierarchy and Caching
% \input{lecture-14}

% % Lecture 15: Direct Mapped Cache Control
% \input{lecture-15}

% % Lecture 16: Associative Cache Control
% \section{Lecture 16: Cache Write Policies and Associative Caches}

\emph{By Dr. Isuru Nawinne}

\subsection{Introduction}

This lecture explores advanced cache design techniques that significantly impact memory system performance. We examine write policies—specifically write-through and write-back strategies—understanding how each handles the critical challenge of maintaining consistency between cache and main memory while balancing performance and complexity. The lecture then progresses to associative cache organizations, from direct-mapped through set-associative to fully-associative designs, revealing how different levels of associativity affect hit rates, access latency, and hardware complexity. Through detailed examples and performance analysis, we discover how modern cache systems make strategic trade-offs between speed, capacity utilization, and implementation cost to achieve optimal memory hierarchy performance.

\subsection{Recap: Write Access in Direct Mapped Cache}

\subsubsection{Write-Through Policy}

\begin{itemize}
\item When a write access occurs, the cache controller determines if it's a hit or miss through tag comparison
\item On a write hit: The block is in cache, update it. The cache copy becomes different from memory (inconsistent)
\item Write-through solution: Always write to both cache and memory simultaneously
\item On a write miss: Stall the CPU, fetch the missing block from memory, update the cache, and write to memory

\subsubsection{Advantages of Write-Through}

\begin{itemize}
\item Simple to implement - straightforward cache controller design
\item Old blocks can be discarded without concern since memory is always up-to-date
\item Can overlap writing and tag comparison operations since corrupted data can be safely discarded on a miss

\subsubsection{Disadvantages of Write-Through}

\begin{itemize}
\item Generates heavy write traffic to memory
\item Every cache write triggers a memory write
\item Bus between cache and memory can become congested
\item Inefficient when programs have many store instructions
\item CPU must stall for 10-100 clock cycles on each memory write

\subsubsection{Write Buffer Solution}

\begin{itemize}
\item A FIFO (First In First Out) queue between cache and memory
\item Cache puts write requests in the buffer instead of directly to memory
\item Memory processes requests from buffer at its own speed
\item Allows CPU to continue without waiting for memory
\item Works well for burst writes (short sequences of writes with gaps between)
\item Limitation: If CPU generates continuous writes, buffer fills up and CPU must still stall

\subsection{Write-Back Policy}

\subsubsection{Basic Concept}

\begin{itemize}
\item Write to cache only, not to memory immediately
\item Allow cache and memory to be inconsistent
\item Write blocks back to memory only when evicted from cache

\subsubsection{Dirty Bit}

\begin{itemize}
\item An additional bit array in cache structure (alongside valid, tag, data)
\item Tracks whether a cache block has been modified
\item Set when block is written to cache
\item Indicates that memory copy is not up-to-date

\subsubsection{Write-Back Operations}

\textbf{On Write Hit:}

\begin{itemize}
\item Simply update the cache entry
\item Set the dirty bit to indicate inconsistency
\item Do not write to memory

\textbf{On Read Miss:}

\begin{itemize}
\item Fetch missing block from memory
\item If old block at that entry is dirty (dirty bit = 1):
\item Write old block back to memory first
\item Then fetch new block and overwrite
\item If old block is not dirty:
\item Directly fetch new block and overwrite

\textbf{On Write Miss:}

\begin{itemize}
\item If old block is dirty:
\item Write old block back to memory
\item Fetch new block from memory
\item Update cache entry only (not memory)

\subsubsection{Advantages of Write-Back}

\begin{itemize}
\item Significantly reduces write traffic to memory
\item More efficient when programs have many write accesses
\item Cache is fast; only writing to cache most of the time
\item Write buffer can be used for evicted dirty blocks

\subsubsection{Disadvantages of Write-Back}

\begin{itemize}
\item More complex cache controller
\item Need to maintain and check dirty bit
\item More hardware required
\item More logic in controller design

\subsubsection{Write-Back Cache Structure}

\begin{itemize}
\item Data array
\item Tag array
\item Valid bit array
\item Dirty bit array (new addition)

\subsection{Cache Performance}

\subsubsection{Average Access Time Formula}

T_avg = Hit Latency + Miss Rate $\times$ Miss Penalty

\textbf{Where:}

\begin{itemize}
\item Hit Latency: Time to determine a hit (always present)
\item Miss Rate: 1 - Hit Rate (fraction of accesses that are misses)
\item Miss Penalty: Time to fetch missing block from memory
\item Can be expressed in absolute time (nanoseconds) or clock cycles

\subsubsection{Example Calculation}

\textbf{Given:}

\begin{itemize}
\item Miss Penalty = 20 CPU cycles
\item Hit Rate = 95% (0.95)
\item Hit Latency = 1 CPU cycle
\item Clock Period = 1 nanosecond (1 GHz)

T_avg = 1 + (1 - 0.95) $\times$ 20 = 1 + 0.05 $\times$ 20 = 2 cycles = 2 nanoseconds

\textbf{If hit rate improves to 99.9\%:}

T_avg = 1 + (1 - 0.999) $\times$ 20 = 1 + 0.001 $\times$ 20 = 1.02 cycles

This shows significant improvement from better hit rate.

\subsubsection{Performance Example Problem}

\textbf{Given:}

\begin{itemize}
\item Program with 36% load/store instructions
\item Ideal CPI = 2 (assuming perfect caches)
\item Instruction cache miss rate = 2%
\item Data cache miss rate = 4%
\item Miss penalty = 100 cycles

\textbf{Calculating Actual CPI:}

\begin{itemize}
\item Stalls from instruction cache misses: I $\times$ 0.02 $\times$ 100 = 2I cycles
\item Stalls from data cache misses: I $\times$ 0.36 $\times$ 0.04 $\times$ 100 = 1.44I cycles
\item Total stall cycles: 3.44I
\item Actual CPI = 2 + 3.44 = 5.44 cycles per instruction

\textbf{Speedup with ideal caches:} 5.44 / 2 = 2.72$\times$

\textbf{CPI with no caches:}

\begin{itemize}
\item Every instruction fetch: 100 cycles
\item 36% need data memory: 0.36 $\times$ 100 = 36 cycles
\item Total CPI = 2 + 100 + 36 = 138 cycles
\item Slowdown without caches: 138 / 5.44 = 25.37$\times$

\subsection{Improving Cache Performance}

\subsubsection{Three Factors to Improve}

\begin{enumerate}
\item Hit Rate - increase the percentage of hits
\item Hit Latency - reduce time to determine hits
\item Miss Penalty - reduce time to fetch missing blocks

\subsubsection{Improving Hit Rate}

\textbf{Method 1: Larger Cache Size}

\begin{itemize}
\item More cache blocks means more index bits
\item Reduces probability of multiple addresses mapping to same index
\item Better exploitation of temporal locality
\item Trade-off: Higher cost (SRAM is expensive, ~$2000 per gigabyte)
\item Trade-off: More chip area required

\subsubsection{Direct Mapped Cache Limitation}

\begin{itemize}
\item Multiple memory blocks can map to same cache index
\item Even with empty cache blocks elsewhere, conflicts cause evictions
\item Temporal locality suggests recently accessed blocks should stay
\item But direct mapping forces eviction even when space is available

\subsection{Fully Associative Cache}

\subsubsection{Concept}

\begin{itemize}
\item Eliminate index field - no fixed mapping
\item A block can be placed anywhere in cache
\item Address divided into: Tag + Offset (no index)
\item Tag is larger since no index bits

\subsubsection{Finding Blocks}

\begin{itemize}
\item Cannot use index to locate block
\item Sequential search is too slow
\item Solution: Parallel tag comparison
\item Compare incoming tag with all stored tags simultaneously
\item Requires one comparator per cache entry

\subsubsection{Implementation}

\begin{itemize}
\item Need duplicate comparator hardware for each entry
\item Practical only for small number of entries (8, 16, 32, 64)
\item As entries increase: more comparators, longer wires, more delays

\subsubsection{Block Placement}

\begin{itemize}
\item Find first available invalid entry
\item Use sequential logic to search for invalid bit
\item Takes more time than direct mapped

\subsubsection{Block Replacement}

When all entries are valid, need replacement policy to choose which block to evict.

\subsubsection{Replacement Policies}

\begin{enumerate}
\item LRU (Least Recently Used) - IDEAL:

\begin{itemize}
\item Evict the block that was used longest ago
\item Best exploits temporal locality
\item Very complex to implement
\item Need to timestamp every access
\item Expensive in hardware

\begin{enumerate}
\item Pseudo-LRU (PLRU):

\begin{itemize}
\item Approximation of LRU
\item Simpler mechanism than true LRU
\item 90-99% of time picks least recently used
\item Better balance of performance and complexity

\begin{enumerate}
\item FIFO (First In First Out):
\item Evict block that entered cache first
\item Very simple implementation
\item Only update when new block fetched (not on every access)
\item Lower likelihood of picking LRU block
\item Used in embedded systems for simplicity and low power

\subsubsection{Fully Associative - Advantages}

\begin{itemize}
\item High utilization of cache space
\item Better hit rate (fewer conflict misses)
\item Can choose replacement policy based on needs

\subsubsection{Fully Associative - Disadvantages}

\begin{itemize}
\item Block placement is slow (increases miss penalty)
\item Higher power consumption
\item Higher cost (more hardware)
\item Parallel tag comparison requires duplicate hardware

\subsection{Set Associative Cache}

\subsubsection{Concept}

\begin{itemize}
\item Combines direct mapped and fully associative approaches
\item Add multiple "ways" - duplicate the tag/valid/data arrays
\item Each index refers to a "set" containing multiple blocks
\item Called "N-way set associative" where N is number of ways

\subsubsection{Two-Way Set Associative}

\begin{itemize}
\item Two copies of tag/valid/data arrays
\item Each index points to a set with 2 blocks
\item Index field selects the set
\item Tag comparison done in parallel within the set
\item Doubles cache capacity compared to direct mapped with same number of sets

\subsubsection{Read Access Process}

\begin{enumerate}
\item Use index to select correct set (via demultiplexer)
\item Extract both stored tags from the set
\item Parallel comparison of both tags with incoming tag
\item Each way has hit status (hit0, hit1)
\item Use encoder to generate select signal for multiplexer
\item Select correct data block based on which way hit
\item Use offset to select correct word within block

\subsubsection{Important Notes}

\begin{itemize}
\item Only one tag can match (each tag identifies unique block)
\item If no tags match, it's a miss
\item More complex hardware than direct mapped
\item Higher hit latency due to encoding and multiplexing delays

\subsection{Associativity Spectrum}

\subsubsection{For an 8-Block Cache, Different Organizations}

1-way set associative (Direct Mapped):

\begin{itemize}
\item 8 entries, 1 way each
\item 3-bit index
\item Each block has fixed location

2-way set associative:

\begin{itemize}
\item 4 entries, 2 ways each
\item 2-bit index
\item Each set can hold 2 different blocks

\begin{figure}[h]
\centering
\includegraphics[width=0.7\textwidth]{img/Memory%20Systems.jpg}
\caption{Memory System}
\end{figure}

4-way set associative:

\begin{itemize}
\item 2 entries, 4 ways each
\item 1-bit index
\item Each set can hold 4 different blocks

8-way set associative (Fully Associative):

\begin{itemize}
\item 1 entry, 8 ways
\item No index field (0 bits)
\item Any block can go anywhere

\subsubsection{Design Considerations}

\begin{itemize}
\item Choice depends on: program behavior, CPU architecture, performance goals, power budget
\item Higher associativity $\rightarrow$ better hit rate
\item Higher associativity $\rightarrow$ higher hit latency
\item Higher associativity $\rightarrow$ more power consumption and cost

\subsection{Associativity Comparison Example}

\subsubsection{Setup}

\begin{itemize}
\item Four-block cache (4 different blocks)
\item Block size = 1 word = 4 bytes
\item 8-bit addresses
\item Compare: Direct Mapped, 2-way Set Associative, Fully Associative (4-way)

\subsubsection{Initial State}

\begin{itemize}
\item All valid bits = 0 (invalid)
\item All tags = 0
\item Data unknown (don't care)

\subsubsection{Tag and Index Sizes}

\begin{itemize}
\item Direct Mapped: 4-bit tag, 2-bit index, 2-bit offset
\item 2-way Set Associative: 5-bit tag, 1-bit index, 2-bit offset
\item Fully Associative: 6-bit tag, 0-bit index, 2-bit offset

\subsubsection{Memory Access Sequence}

\textbf{Access 1: Block Address 0}

\begin{itemize}
\item All three caches: MISS (cold miss - first time accessed)
\item All valid bits were 0
\item Fetch from memory, update tag, set valid bit

\textbf{Score:} Direct Mapped: 0 hits, 1 miss | 2-way: 0 hits, 1 miss | Fully: 0 hits, 1 miss

\textbf{Access 2: Block Address 8}

\begin{itemize}
\item All three caches: MISS (cold miss - first time accessed)
\item Tags don't match existing entries
\item Fetch from memory, place in cache

\textbf{Score:} Direct Mapped: 0 hits, 2 misses | 2-way: 0 hits, 2 misses | Fully: 0 hits, 2 misses

\textbf{Access 3: Block Address 0 (repeated)}

\begin{itemize}
\item Direct Mapped: MISS (conflict miss - block 8 overwrote block 0 at same index)
\item 2-way Set Associative: HIT (both blocks 0 and 8 fit in same set)
\item Fully Associative: HIT (both blocks present)
\item Demonstrates advantage of associativity

\textbf{Score:} Direct Mapped: 0 hits, 3 misses | 2-way: 1 hit, 2 misses | Fully: 1 hit, 2 misses

\textbf{Access 4: Block Address 6}

\begin{itemize}
\item All three: MISS (cold miss)
\item 2-way: Set full, need replacement
\item LRU replacement: evict block 8 (least recently used)
\item FIFO replacement: would evict block 0 (first in)
\item Fully Associative: Still has empty space

\textbf{Score:} Direct Mapped: 0 hits, 4 misses | 2-way: 1 hit, 3 misses | Fully: 1 hit, 3 misses

\textbf{Access 5: Block Address 8 (repeated)}

\begin{itemize}
\item Direct Mapped: MISS (conflict miss - keeps conflicting at index 0)
\item 2-way: MISS (conflict miss - block 8 was evicted by block 6)
\item Fully Associative: HIT (block 8 still in cache)

\subsubsection{Final Score}

\begin{itemize}
\item Direct Mapped: 0 hits, 5 misses (all misses after cold misses)
\item 2-way Set Associative: 1 hit, 4 misses (one conflict miss)
\item Fully Associative: 2 hits, 3 misses (only cold misses)

\subsubsection{Types of Misses}

\begin{enumerate}
\item Cold Misses: First access to address (unavoidable)
\item Conflict Misses: Block evicted due to mapping, accessed again later

\subsubsection{Key Observations}

\begin{itemize}
\item Higher associativity reduces conflict misses
\item Fully associative eliminates conflict misses (only cold misses remain)
\item But higher associativity increases hit latency and cost

\subsection{Trade-Offs Summary}

\subsubsection{Hit Rate}

\begin{itemize}
\item Increases with higher associativity
\item Direct mapped has most conflict misses
\item Fully associative has only cold misses

\subsubsection{Hit Latency}

\begin{itemize}
\item Increases with higher associativity
\item More comparators, encoders, multiplexers add delay
\item Direct mapped is fastest

\subsubsection{Power and Cost}

\begin{itemize}
\item Increases with higher associativity
\item More hardware for parallel comparison
\item More complex control logic

\subsubsection{Design Decision Factors}

\begin{itemize}
\item Application requirements
\item Performance goals
\item Power budget
\item Cost constraints
\item Embedded systems often use lower associativity (FIFO replacement)
\item High-performance systems use higher associativity (PLRU replacement)

\subsection{Key Takeaways}

\begin{enumerate}
\item \textbf{Write policies} manage cache-memory consistency:
\item Write-through: Simple but generates heavy memory traffic
\item Write-back: More efficient but requires dirty bit tracking
\item \textbf{Write buffers} improve write-through performance by decoupling cache and memory writes
\item \textbf{Cache performance} depends on three factors: hit rate, hit latency, and miss penalty
\item \textbf{Associativity spectrum} ranges from direct-mapped (1-way) to fully associative (N-way)
\item \textbf{Higher associativity} reduces conflict misses and improves hit rate but increases complexity
\item \textbf{Set-associative caches} balance the trade-offs between direct-mapped and fully associative designs
\item \textbf{Replacement policies} (LRU, PLRU, FIFO) determine which block to evict in associative caches
\item \textbf{Design decisions} must balance performance, power consumption, cost, and complexity
\item \textbf{Real-world caches} use different associativity levels based on application requirements
\end{enumerate}

10. \textbf{Performance analysis} shows that even small improvements in hit rate significantly reduce average access time

\subsection{Summary}

This lecture examined two critical aspects of cache design: write policies and associativity. Write-through and write-back policies each offer distinct trade-offs between simplicity and efficiency, with write buffers providing a middle ground that improves performance without excessive complexity. The exploration of associative cache organizations revealed how different levels of associativity—from direct-mapped through set-associative to fully-associative—affect hit rates, access latency, and hardware complexity. Through detailed performance analysis and practical examples, we discovered that while higher associativity generally improves hit rates by reducing conflict misses, it comes at the cost of increased hit latency, power consumption, and implementation complexity. Modern cache systems carefully balance these competing factors, with set-associative designs emerging as an effective compromise that captures most of the benefits of full associativity while maintaining reasonable complexity. Understanding these design trade-offs is essential for optimizing memory hierarchy performance in real-world computer systems.


% % Lecture 17: Multi-Level Caching
% \input{lecture-17}

% % Lecture 18: Virtual Memory
% \section{Lecture 18: Virtual Memory}

\emph{By Dr. Isuru Nawinne}

\subsection{Introduction}

Virtual memory represents one of the most elegant abstractions in computer architecture, creating a layer between physical memory hardware and the memory view presented to programs. This lecture explores how virtual memory enables programs to use more memory than physically available by treating main memory as a cache for disk storage, supports safe execution of multiple concurrent programs through address space isolation, and provides memory protection mechanisms preventing programs from corrupting each other's data. We examine page tables, translation lookaside buffers (TLBs), page faults, and the critical design decisions that make virtual memory both practical and performant despite the enormous speed gap between RAM and disk storage.

\subsection{Introduction to Virtual Memory}

Virtual memory allows programs to use more memory than physically available by using main memory as a cache for secondary storage.

\subsubsection{Key Purposes of Virtual Memory}

\begin{enumerate}
\item \textbf{Allow programs to use more memory than actually available}
\item \textbf{Support multiple programs running simultaneously on a CPU}
\item \textbf{Enable safe and efficient memory sharing between programs}
\item \textbf{Ensure programs only access their allocated memory}

\subsection{CPU Word Size and Address Space}

The relationship between CPU word size and addressable memory determines the maximum amount of memory that can be addressed.

\subsubsection{Address Space by CPU Word Sizerd Size}

\paragraph{8-bit CPU}

\begin{itemize}
\item \textbf{Maximum addressable memory}: 256 bytes (2^8)

\paragraph{16-bit CPU}

\begin{itemize}
\item \textbf{Maximum addressable memory}: 64 kilobytes (2^16)

\paragraph{32-bit CPU}

\begin{itemize}
\item \textbf{Maximum addressable memory}: 4 gigabytes (2^32)
\item Became mainstream in early 1980s
\item Was replaced when systems started reaching 4 GB memory limit

\paragraph{64-bit CPU}

\begin{itemize}
\item \textbf{Maximum addressable memory}: 16 exabytes (2^64)
\item About 16 million gigabytes
\item Current mainstream word size
\item Became mainstream around 2002-2003

\subsubsection{Historical Pattern}

\begin{itemize}
\item Maximum address space sizes were always much larger than commonly used RAM sizes
\item Architectures were replaced when high-end systems started reaching the address space limits
\item Personal computers typically had much less memory than the theoretical maximum

\subsection{Virtual vs Physical Addresses}

\subsubsection{Virtual Address}

\begin{itemize}
\item \textbf{Address generated by CPU}
\item Refers to entire theoretical address space
\item CPU thinks it has access to full address space
\item In 64-bit CPU: can address up to 16 exabytes

\subsubsection{Physical Address}

\begin{itemize}
\item \textbf{Actual address in real memory (RAM)}
\item Much smaller range than virtual addresses
\item Typical modern RAM: 8-16 GB (much less than 16 exabytes)

\subsubsection{Address Translation}

\begin{itemize}
\item Virtual addresses must be translated to physical addresses
\item Translation required every time memory is accessed
\item Main mechanism for making virtual memory work

\subsection{Memory Hierarchy with Virtual Memory}

Complete hierarchy from top to bottom:

\begin{enumerate}
\item \textbf{CPU} (generates virtual addresses, thinks memory is large and fast)
\item \textbf{Cache} (virtually or physically addressed)
\item \textbf{Main Memory} (acts as cache for secondary storage)
\item \textbf{Secondary Storage/Disk} (contains all pages)

CPU accesses cache directly. Main memory acts as cache for disk, not just a second level cache - requires additional mechanisms.

\subsection{Terminology}

\subsubsection{CPU Level}

\begin{itemize}
\item \textbf{Accesses}: Words (1, 4, or 8 bytes)
\item Hit/Miss terminology used

\subsubsection{Cache Level}

\begin{itemize}
\item \textbf{Transfers}: Blocks (16-256 bytes typically)
\item Hit/Miss terminology used

\subsubsection{Memory Level}

\begin{itemize}
\item \textbf{Transfers}: Pages (1 KB to 64 KB typically)
\item \textbf{Page Hit}: Page is present in memory
\item \textbf{Page Fault}: Page is not present in memory (not "miss")

\subsection{Access Latencies}

Understanding the latency differences is crucial for virtual memory design:

\begin{itemize}
\item \textbf{Cache Hit}: Under 1 cycle
\item \textbf{Cache Miss} (accessing main memory): 10-100 cycles
\item \textbf{Page Fault} (accessing disk): ~1 million cycles
\item Extremely large penalty
\item Influences design decisions significantly
\item Page faults handled in software by OS due to large penalty

![Access Latencies](img/Virtual_mem_cycles.jpg)

\subsection{Virtual and Physical Address Structure}

\subsubsection{Example with 32-bit Addressesdresses}

\paragraph{Virtual Address (32 bits)}

\begin{itemize}
\item \textbf{Virtual Page Number}: 22 bits (most significant)
\item \textbf{Page Offset}: 10 bits (least significant)
\item \textbf{Virtual address space}: 4 GB
\item \textbf{Number of virtual pages}: 2^22 pages
\item \textbf{Page size}: 2^10 = 1 KB

\paragraph{Physical Address (28 bits)}

\begin{itemize}
\item \textbf{Physical Page Number (Frame Number)}: 18 bits (most significant)
\item \textbf{Page Offset}: 10 bits (least significant)
\item \textbf{Physical address space}: 256 MB
\item \textbf{Number of frames}: 2^18 frames
\item \textbf{Page size}: 1 KB (same as virtual)

\subsubsection{Key Points}

\begin{itemize}
\item Page offset has same number of bits in virtual and physical addresses
\item Physical address space is smaller than virtual address space
\item Memory contains "frames" where pages can be placed
\item Frame = slot in memory that can hold a page

\subsection{Supporting Multiple Programs}

Multiple programs can run simultaneously by sharing physical memory:

\subsubsection{Each Program}

\begin{itemize}
\item Has its own virtual address space
\item Thinks it has entire memory to itself
\item CPU switches between programs quickly
\item Creates impression of simultaneous execution

\subsubsection{Memory Sharing}

\begin{itemize}
\item Physical memory contains active pages from all running programs
\item Each program's virtual pages map to different physical frames
\item Operating system ensures programs only access their own memory

\subsubsection{Example}

\begin{itemize}
\item Program 1 virtual address space: 8 virtual pages
\item Program 2 virtual address space: 8 virtual pages
\item Physical memory: Only 4 frames available
\item Active pages from both programs share the 4 frames
\item Same virtual page number from different programs can map to different physical frames

\subsection{Page Table}

The page table is a data structure stored in memory that contains address translations.

\subsubsection{Purpose}

\begin{itemize}
\item Stores virtual-to-physical address translations
\item One page table per program
\item Contains entries for ALL virtual pages (not just active ones)

\subsubsection{Page Table Entry Contents}

\begin{enumerate}
\item \textbf{Physical Page Number} (main component)
\item \textbf{Valid Bit}: Is the page currently in memory?
\item 1 = Page is in memory (translation valid)
\item 0 = Page not in memory (page fault)
\item \textbf{Dirty Bit}: Has page been modified?
\item 1 = Page modified, inconsistent with disk
\item 0 = Page not modified, consistent with disk
\item \textbf{Additional bits}: Access permissions, memory protection status

\subsubsection{Finding Page Table}

\begin{itemize}
\item Page tables stored at fixed locations in memory
\item \textbf{Page Table Base Register (PTBR)}: Special CPU register storing starting address of active page table
\item When CPU switches programs, OS updates PTBR to point to correct page table

\subsection{Address Translation Process}

![Address Translation Process](img/Virtual_Mem_Translation.jpg)

Steps to access memory:

\begin{enumerate}
\item \textbf{CPU generates virtual address} (virtual page number + page offset)
\item \textbf{Access page table} using PTBR + virtual page number as index
\item \textbf{Read page table entry}:
\item If valid bit = 0: Page fault (handled by OS)
\item If valid bit = 1: Read physical page number
\item \textbf{Construct physical address}: Physical page number + page offset
\item \textbf{Access physical memory} with physical address
\item \textbf{Return data to CPU}

\subsubsection{Memory Accesses Required}

\begin{itemize}
\item One access for page table
\item One access for actual data
\item \textbf{Total}: Two memory accesses per data access

\subsection{Page Table Size Calculation}

\subsubsection{Example: 4 GB Virtual, 1 GB Physical, 1 KB PagesKB Pages}

\paragraph{Number of Entries}

\begin{itemize}
\item Virtual address: 32 bits
\item Page offset: 10 bits (for 1 KB pages)
\item Virtual page number: 22 bits
\item \textbf{Number of entries}: 2^22 = ~4 million entries

\paragraph{Entry Size}

\begin{itemize}
\item Physical address: 30 bits (for 1 GB)
\item Page offset: 10 bits
\item Physical page number: 20 bits
\item Valid bit: 1 bit
\item Dirty bit: 1 bit
\item Total needed: 22 bits
\item Actual storage: 32 bits (word-aligned)
\item \textbf{Size per entry}: 4 bytes

\paragraph{Total Page Table Size}

\begin{itemize}
\item 4 bytes $\times$ 2^22 entries = \textbf{16 MB}
\item Significant memory overhead for each program

\subsection{Write Policy for Virtual Memory}

\subsubsection{Write-Through: NOT USED}

\begin{itemize}
\item Would require writing to disk on every write
\item 1 million cycle penalty unacceptable
\item Not a good design decision

\subsubsection{Write-Back: USED (Standard Policy)}

\begin{itemize}
\item Writes only update memory
\item Dirty bit tracks modified pages
\item Only write to disk when:
\item Page is evicted from memory
\item Page's dirty bit is 1
\item Minimizes disk accesses

\subsection{Placement Policy}

\subsubsection{Fully Associative Placement}

\begin{itemize}
\item Any page from disk can go to any frame in memory
\item Memory treated as one large set containing all frames
\item No direct mapping or set restrictions
\item Maximizes flexibility in page placement
\item Reduces page faults

\subsubsection{Why Fully Associative?}

\begin{itemize}
\item Minimizes page faults (primary goal)
\item Large page fault penalty (1 million cycles) justifies complexity
\item Different from cache (doesn't use tag comparators in memory)
\item Address translation through page table provides necessary mechanism

\subsection{Page Fault Handling}

\subsubsection{What Operating System Must Do Must Do}

\paragraph{1. Fetch Missing Page}

\begin{itemize}
\item Access disk to retrieve page
\item OS must know disk location of page
\item OS maintains data structures tracking page locations

\paragraph{2. Find Unused Frame}

\begin{itemize}
\item OS tracks which frames are currently used
\item Can determine this through page tables
\item If unused frame exists: Place page in unused frame

\paragraph{3. If Memory Full (No Unused Frames)}

\begin{itemize}
\item Select active page to replace using replacement policy
\item Common replacement policies:
\item Least Recently Used (LRU)
\item Pseudo-LRU (PLRU)
\item First-In-First-Out (FIFO)
\item Least Frequently Used (LFU)
\item Goal: Exploit temporal locality (keep recently/frequently used pages)

\paragraph{4. Check Dirty Bit of Page to be Replaced}

\begin{itemize}
\item If dirty bit = 1: Write page back to disk before replacement
\item If dirty bit = 0: Can directly overwrite (data consistent with disk)
\item Prevents data loss

\paragraph{5. Update Data Structures}

\begin{itemize}
\item Update page table entry for new page
\item Update page table entry for replaced page (set valid = 0)
\item Place fetched page in frame

\subsubsection{Optimization}

\begin{itemize}
\item Many operations can occur in parallel during disk fetch
\item While fetching data, OS can determine placement and handle replacement
\item Use buffers for write-back operations

\subsubsection{Why Software Handling?}

\begin{itemize}
\item 1 million cycle penalty is so large that software overhead is negligible
\item Complex replacement policies better suited to software
\item Hardware optimization doesn't provide significant benefit

\subsection{Translation Lookaside Buffer (TLB)}

![TLB](img/Virtual_mem_TLB.jpg)

\subsubsection{Purpose}

\begin{itemize}
\item Avoid accessing memory twice for every data access
\item Act as cache for page table entries
\item Reduce address translation overhead

\subsubsection{What is TLB?}

\begin{itemize}
\item Hardware cache specifically for page table entries
\item Stores recently used address translations
\item Based on locality of page table entry accesses
\item Exploits temporal and spatial locality of page accesses

\subsubsection{TLB Entry Structure}

\begin{itemize}
\item \textbf{Tag}: Virtual address tag (or physical address tag)
\item \textbf{Physical Page Number}: The address translation
\item \textbf{Valid Bit}: Is this TLB entry valid?
\item Different from page table valid bit
\item Indicates if TLB entry contains valid translation
\item \textbf{Dirty Bit}: Same meaning as in page table

\subsubsection{TLB Parametersrameters}

\paragraph{Size}

\begin{itemize}
\item \textbf{16-512 page table entries} (typical range)

\paragraph{Block Size}

\begin{itemize}
\item \textbf{1-2 address translations}
\item Small blocks because spatial locality between pages is larger
\item Adjacent pages not as closely related as adjacent cache blocks

\paragraph{Placement Policy}

\begin{itemize}
\item Fully associative or set associative
\item Fully associative for smaller TLBs (~16 entries)
\item Set associative for larger TLBs
\item Goal: Keep miss rate below 1%

\paragraph{Hit Latency}

\begin{itemize}
\item \textbf{Much less than 1 cycle}

\paragraph{Miss Penalty}

\begin{itemize}
\item \textbf{10-100 cycles} (memory access required)

\subsubsection{TLB Operationperation}

\paragraph{Hit}

\begin{itemize}
\item Address translation available in TLB
\item Use translation directly without accessing memory
\item Only one memory access needed (for data)

\paragraph{Miss}

\begin{itemize}
\item Translation not in TLB
\item Must access page table in memory
\item Total: Two memory accesses (page table + data)

\subsubsection{Why Low Miss Rate Essential?}

\begin{itemize}
\item TLB misses double memory access time
\item Must access page table (10-100 cycles) then data
\item Miss rate typically kept below 1%
\item > 99% of translations served by TLB

\subsection{Complete Memory Access with TLB}

Two different approaches for handling memory access with TLB:

\subsection{Approach 1: Virtually Addressed Cache}

\subsubsection{Process}

\begin{enumerate}
\item \textbf{CPU generates virtual address}
\item \textbf{Access cache with virtual address} (parallel with TLB)
\item \textbf{Cache Hit}: Return data to CPU immediately
\item \textbf{Cache Miss}:

   a. \textbf{Check TLB for address translation}

   b. \textbf{TLB Hit}:

\begin{itemize}
\item Get physical address
\item Access memory with physical address
\item Fetch missing block
\item Update cache
\item Send word to CPU

   c. \textbf{TLB Miss}:

\begin{itemize}
\item Access page table in memory
\item \textbf{Page Hit}:
\item Get translation
\item Access memory for data
\item Update TLB
\item Update cache
\item Send word to CPU
\item \textbf{Page Fault}:
\item OS accesses disk
\item Fetch missing page
\item Find unused frame or replace page
\item If replaced page dirty: write back
\item Update page table
\item Update TLB
\item Update cache
\item Send word to CPU

\subsubsection{Advantage}

\begin{itemize}
\item TLB access overlapped with cache access
\item Both happen in parallel
\item No additional latency for TLB access on cache hit

\subsection{Approach 2: Physically Addressed Cache}

\subsubsection{Process}

\begin{enumerate}
\item \textbf{CPU generates virtual address}
\item \textbf{Access TLB for translation first}
\item \textbf{TLB Hit}:

   a. Get physical address

   b. \textbf{Access cache with physical address}

   c. \textbf{Cache Hit}: Return data to CPU

   d. \textbf{Cache Miss}:

\begin{itemize}
\item Access memory with physical address
\item Fetch missing block
\item Update cache
\item Send word to CPU

\begin{enumerate}
\item \textbf{TLB Miss}:

   a. Access page table in memory

   b. \textbf{Page Hit}:

\begin{itemize}
\item Get translation
\item Update TLB
\item Access cache with physical address
\item If cache hit: return data
\item If cache miss: fetch from memory, update cache, return data

   c. \textbf{Page Fault}:

\begin{itemize}
\item OS handles as described above
\item Update page table, TLB, cache
\item Return data to CPU

\subsubsection{Advantage}

\begin{itemize}
\item Cache physically indexed and tagged
\item Simpler cache design
\item No aliasing issues

\subsubsection{Key Difference}

\begin{itemize}
\item \textbf{Approach 1}: Cache uses virtual addresses, TLB access parallel
\item \textbf{Approach 2}: Cache uses physical addresses, TLB access sequential

Both approaches are valid, and the choice depends on cache indexing method (virtual vs physical).

\subsection{Key Takeaways}

\begin{enumerate}
\item Virtual memory provides memory abstraction and protection
\item Address translation is fundamental to virtual memory operation
\item Page tables map virtual addresses to physical addresses
\item TLB caches translations to avoid double memory access
\item Page faults are extremely expensive (~1 million cycles)
\item Write-back policy is essential for virtual memory
\item Fully associative placement minimizes page faults
\item Multiple programs can safely share physical memory
\item OS handles page faults in software
\end{enumerate}

10. Virtual memory enables modern multitasking operating systems

\subsection{Summary}

Virtual memory represents a crucial abstraction in modern computing, enabling efficient and safe memory management across multiple concurrent programs while providing each program with the illusion of abundant, dedicated memory resources.


% \chapter{Advanced Topics}

% % Lecture 19: Multiprocessors
% \section{Lecture 19: Multiprocessors}

\emph{By Dr. Isuru Nawinne}

\subsection{Introduction}

Multiprocessor systems represent a fundamental paradigm shift in computer architecture, using multiple processors on the same chip to execute multiple programs or threads simultaneously when traditional performance improvement techniques—clock frequency scaling and instruction-level parallelism—reached physical and practical limits. This lecture explores the evolution toward multiprocessor architectures driven by power walls and parallelism walls, examines the critical challenge of cache coherence that arises when multiple processors maintain private caches of shared memory, and analyzes solutions including bus snooping protocols like MESI and scalable directory-based coherence schemes. We compare architectural organizations from uniform memory access (UMA) to non-uniform memory access (NUMA), understanding how different designs balance simplicity, performance, and scalability for systems ranging from dual-core smartphones to thousand-processor supercomputers.

\subsection{Introduction to Multiprocessors}

Multiprocessor systems address performance limitations encountered with single processor systems by employing multiple processors on the same chip to execute multiple programs or threads simultaneously.

\subsection{Performance Evolution Background}

\subsubsection{Historical Performance Improvementsvements}

\paragraph{Early Methods: Clock Frequency Scaling}

\textbf{Approach}:

\begin{itemize}
\item Increasing clock frequency (reducing clock cycle time)
\item Goal: Spend less time per instruction

\textbf{Limitations Encountered}:

\begin{itemize}
\item Hit barrier at ~4 GHz: Power wall problem
\item Excessive power dissipation caused overheating
\item Cooling became inadequate
\item Could not sustainably increase frequency further

\paragraph{Instruction Level Parallelism (ILP)}

\textbf{Techniques}:

\begin{itemize}
\item \textbf{Pipelining}: Process multiple instructions simultaneously
\item Utilize different hardware components at same time
\item Example: Execute one instruction while fetching another
\item \textbf{Advanced techniques}: Multiple issue, out-of-order execution
\item Exploit parallelism inherent in programs

\textbf{Limitations Encountered}:

\begin{itemize}
\item Programs contain limited inherent parallelism
\item Dependencies prevent unlimited parallel execution
\item Hit "parallelism wall"
\item Can't exploit more parallelism beyond program's inherent limits

\subsubsection{Moore's Law Context}

\textbf{Observation}:

\begin{itemize}
\item Number of transistors doubles every 2 years
\item Technology improves, more transistors available

\textbf{Question}: How to use abundant transistors?

\textbf{Solution}: Multiple processors on same chip

\subsection{Multiprocessor Approach}

\subsubsection{Key Characteristics}

\begin{itemize}
\item Multiple processor cores on same chip
\item Execute multiple instruction streams simultaneously
\item Run multiple programs/threads in real time (true parallelism)
\item Different from single processor illusion of parallelism

\subsubsection{Terminology}

\begin{itemize}
\item \textbf{Processing Elements (PE)}: Common term for individual processors
\item Each PE is a complete CPU with fetch, decode, execute units

\subsubsection{Key Problem: Communication Between Processors}

\begin{itemize}
\item Multiple processors executing simultaneously
\item Programs often need to communicate/share data
\item Splitting programs into threads requires coordination
\item Communication is central design challenge

\subsection{Shared Memory Multiprocessors (SMM)}

![Shared Memory Multiprocessors](img/Multiprocessors_SSM.jpg)

\subsubsection{Most Common Approach}

\textbf{Architecture}:

\begin{itemize}
\item Communication through shared memory
\item All processors access same physical address space
\item Memory device connected via common bus/interconnect

\subsubsection{Operating System Role}

\textbf{Responsibilities}:

\begin{itemize}
\item OS code stored in shared memory
\item OS shared between all processors
\item Manages memory access arbitration
\item Performs workload balancing
\item Ensures processors access only authorized memory portions

\subsubsection{Workload Balancing}

\textbf{Purpose}:

\begin{itemize}
\item OS distributes tasks among processors
\item Goal: All processors working in parallel
\item Avoid idle processors
\item Maximize overall system utilization

\subsection{Memory Contention Problem}

\subsubsection{Inherent Issue}

\textbf{Challenge}:

\begin{itemize}
\item Multiple processors accessing same memory device
\item Competition for memory access
\item Processors must wait for memory availability
\item Synchronization overhead
\item Access time increases with contention

\subsubsection{Effect on Performance}

\textbf{Bottleneck}:

\begin{itemize}
\item Memory becomes bottleneck
\item Bus connects all processors to memory
\item If one processor using memory, others must wait
\item Can take hundreds of cycles
\item Limits scalability

\subsection{Uniform Memory Access (UMA)}

![Uniform Memory Access (UMA)](img/Multiprocessors_NVM.jpg)

\subsubsection{Definition}

\textbf{Characteristics}:

\begin{itemize}
\item Each processor sees memory in exact same way
\item Same average memory access time for all processors
\item Access time independent of which processor is accessing
\item By design, no difference in access time (ignoring contention)

\subsubsection{Also Known As}

\begin{itemize}
\item \textbf{Symmetric Multiprocessors (SMP)}
\item Both terms used interchangeably

\subsubsection{Key Properties}

\begin{itemize}
\item Shared address space
\item Uniform view of memory by all processors
\item All processors experience same average latency

\subsection{Solution to Contention: Caches}

\subsubsection{Using Local Caches}

\textbf{Approach}:

\begin{itemize}
\item Each processor has private cache
\item Based on locality principles (temporal and spatial)
\item Most memory accesses served at cache level
\item Only small percentage (misses) go to main memory
\item Reduces bus/memory contention significantly

\subsubsection{Benefits}

\begin{itemize}
\item Exploits locality in programs
\item Minimizes memory accesses
\item Reduces bottleneck effect
\item Allows better scalability

\subsubsection{New Problem: Cache Coherence}

\begin{itemize}
\item Shared data blocks can be in multiple caches
\item Updates in one cache not automatically reflected in others
\item Need mechanism to maintain consistency

\subsection{Cache Coherence Problem}

\subsubsection{The Issue}

\textbf{Scenario}:

\begin{itemize}
\item Multiple caches have copies of same data block
\item One processor writes to that block
\item Other caches have stale (old) data
\item Processors see different values for same address
\item Data becomes incoherent

\subsubsection{Example Sequence}

\begin{enumerate}
\item \textbf{PE1 reads X (value = 1)} $\rightarrow$ Cached in PE1
\item \textbf{PE2 reads X (value = 1)} $\rightarrow$ Cached in PE2
\item \textbf{PE1 writes X = 0} $\rightarrow$ PE1 cache updated
\item Memory may or may not be updated (depends on write policy)
\item PE2 still sees X = 1 (stale data)
\item \textbf{Inconsistency}: Same address, different values

\subsubsection{With Write-Through Policy}

\begin{itemize}
\item PE1 writes X = 0 $\rightarrow$ Cache and memory updated
\item Memory has correct value
\item But PE2 cache still has old value (X = 1)
\item Coherence still lost

\subsubsection{With Write-Back Policy}

\begin{itemize}
\item PE1 writes X = 0 $\rightarrow$ Only cache updated
\item Memory still has old value (X = 1)
\item PE2 cache still has old value (X = 1)
\item Both memory and PE2 incoherent with PE1

\subsubsection{Requirement}

\begin{itemize}
\item Cache coherence MUST be maintained
\item Otherwise parallel programs execute incorrectly
\item Get wrong results
\item Latest updates must be visible to all processors

\subsection{Bus Snooping}

Common technique for cache coherence in SMP systems.

![Bus Snooping](img/Multiprocessors_bus.jpg)

\subsubsection{What is Bus Snooping?}

\textbf{Mechanism}:

\begin{itemize}
\item Dedicated bus for coherency control: \textbf{Snoop bus}
\item Sole purpose: Control coherency of cache data
\item Separate from memory bus
\item Cache controllers communicate through snoop bus

\subsubsection{How It Works}

\begin{enumerate}
\item Cache controller performs write to address
\item Broadcasts address information on snoop bus
\item All cache controllers listen to snoop bus
\item Controllers check if they have same address cached
\item If yes, take action based on protocol

\subsubsection{Key Feature}

\begin{itemize}
\item All caches monitor (snoop on) the bus
\item Detect writes by other processors
\item Take appropriate action to maintain coherence

\subsection{Write Invalidate Protocol}

\subsubsection{Approach}

\begin{itemize}
\item When write detected, invalidate own copy
\item Group of protocols using this approach
\item Most common and easiest to implement

\subsubsection{Mechanism}

\paragraph{On Write by Processor}

\begin{enumerate}
\item Update own cache
\item Broadcast write address on snoop bus

\paragraph{On Receiving Write Broadcast}

\begin{enumerate}
\item Check if same address in own cache
\item If yes: Mark block as INVALID (clear valid bit)
\item Next access will be miss

\subsubsection{With Write-Through Policy}

\begin{itemize}
\item Memory always has up-to-date value
\item On miss after invalidation: Fetch from memory
\item Straightforward implementation

\subsubsection{With Write-Back Policy}

\textbf{Challenge}:

\begin{itemize}
\item Only writing cache has up-to-date value
\item Memory has stale value
\item On miss after invalidation: Cannot fetch from memory

\textbf{Solution: Snoop Read}:

\begin{itemize}
\item Cache with invalid block places snoop read request on bus
\item Cache controllers listen to snoop read
\item Controller with valid up-to-date copy responds
\item Supplies data through snoop bus
\item More efficient than going to memory
\item Avoids slow memory access

\subsubsection{Complexity}

\textbf{Trade-offs}:

\begin{itemize}
\item More complex cache controller
\item Snoop bus needs to carry data and addresses
\item More hardware required
\item Higher power consumption
\item But better performance (less memory traffic)

\subsection{Write Update Protocol}

\subsubsection{Alternative Approach}

\textbf{Concept}:

\begin{itemize}
\item Update own copy instead of invalidating
\item Also called Write Broadcast
\item Different action when write detected

\subsubsection{Mechanism}

\paragraph{On Write by Processor}

\begin{enumerate}
\item Update own cache
\item Broadcast BOTH address AND data on snoop bus

\paragraph{On Receiving Write Broadcast}

\begin{enumerate}
\item Check if same address in own cache
\item If yes: Update own copy with new data
\item Keep block VALID

\subsubsection{Benefits}

\begin{itemize}
\item No miss on next access to same address
\item Data already updated in all caches
\item Don't need extra read operation
\item Simpler cache controller (no snoop read needed)

\subsubsection{Costs}

\begin{itemize}
\item Snoop bus must carry data (wider bus)
\item More hardware on snoop bus
\item Higher power consumption
\item More bus traffic

\subsubsection{Comparison}

\begin{itemize}
\item Simpler than write invalidate with write-back
\item Fewer cache misses
\item Higher bus bandwidth requirement

\subsection{Real Protocol Implementations}

\subsubsection{Historical Protocols}

\paragraph{Write Once Protocol}

\begin{itemize}
\item \textbf{Type}: Write invalidate
\item \textbf{Write policy}: Write-through on first write, write-back after
\item One of first bus snooping protocols

\paragraph{Synapse N+1 Protocol}

\begin{itemize}
\item \textbf{Type}: Write invalidate
\item \textbf{Write policy}: Write-back
\item Early implementation

\paragraph{Berkeley Protocol}

\begin{itemize}
\item \textbf{Type}: Write invalidate
\item \textbf{Write policy}: Write-back
\item Used in Berkeley SPUR processor

\paragraph{Illinois Protocol (MESI)}

\begin{itemize}
\item \textbf{Type}: Write invalidate
\item \textbf{Write policy}: Write-back
\item Used in SGI Power and Challenge systems
\item Very popular, widely adopted

\paragraph{Firefly Protocol}

\begin{itemize}
\item \textbf{Type}: Write update
\item \textbf{Write policy}: Mixed (write-back for private data, write-through for shared data)
\item Used in DEC Firefly and Sun SPARC systems

\subsubsection{Most Common Combination}

\begin{itemize}
\item Write invalidate protocols
\item Write-back policy
\item Reduces memory accesses (expensive in terms of time)
\item Easier to implement than write update
\item Good balance of performance and complexity

\subsection{MESI Protocol Details}

Named after four states: \textbf{Modified, Exclusive, Shared, Invalid}

![MESI](img/Multiprocessors_mesi.jpg)

Most popular cache coherency protocol, used in Intel Pentium and IBM PowerPC processors.

\subsubsection{Four Block States (Requires 2 Bits)}

\paragraph{1. INVALID (I)}

\begin{itemize}
\item Data not valid
\item Block cannot be used
\item Must fetch from elsewhere

\paragraph{2. SHARED (S)}

\begin{itemize}
\item Multiple caches have copies of this block
\item All copies have same value
\item Value consistent with memory
\item Memory has up-to-date value

\paragraph{3. EXCLUSIVE (E)}

\begin{itemize}
\item Only cached copy in entire system
\item No other cache has this block
\item Value consistent with memory
\item Memory has up-to-date value

\paragraph{4. MODIFIED (M)}

\begin{itemize}
\item Only cached copy in system
\item Value INCONSISTENT with memory
\item This cache has most recent value
\item Memory has stale value
\item Block is "dirty"

\subsection{MESI Protocol State Transitions}

\subsubsection{Example with PE1, PE2, PE3}

\textbf{Initial State}: Variable X = 1 in memory, all cache entries invalid

\paragraph{Step 1: PE1 Reads X}

\textbf{Actions}:

\begin{itemize}
\item Check other caches (snoop read request)
\item No other cache has X
\item Fetch from memory
\item \textbf{State transition}: Invalid $\rightarrow$ Exclusive

\textbf{Result}:

\begin{itemize}
\item PE1: X = 1 (Exclusive)

\paragraph{Step 2: PE3 Reads X}

\textbf{Actions}:

\begin{itemize}
\item Check other caches (snoop read request)
\item PE1 responds (has Exclusive copy)
\item PE1 supplies data to PE3

\textbf{State transitions}:

\begin{itemize}
\item PE1: Exclusive $\rightarrow$ Shared
\item PE3: Invalid $\rightarrow$ Shared

\textbf{Result}:

\begin{itemize}
\item Both PE1 and PE3: X = 1 (Shared)
\item Consistent with memory

\paragraph{Step 3: PE3 Writes X = 0}

\textbf{Actions}:

\begin{itemize}
\item Block in PE3 was Shared
\item Update local cache
\item Broadcast invalidate on snoop bus
\item \textbf{State transition}: Shared $\rightarrow$ Modified

\textbf{Result}:

\begin{itemize}
\item PE3: X = 0 (Modified)
\item PE1 receives invalidate:
\item State transition: Shared $\rightarrow$ Invalid
\item PE1: X = ? (Invalid)
\item Memory still has X = 1 (stale)

\paragraph{Step 4: PE1 Reads X}

\textbf{Actions}:

\begin{itemize}
\item Block in PE1 is Invalid (tag matches but invalid)
\item Place snoop read request on bus
\item PE3 has Modified copy (most up-to-date)
\item PE3 responds to snoop read:
\item Supplies data to PE1 through snoop bus
\item Writes back to memory
\item State transition: Modified $\rightarrow$ Shared
\item PE1 receives data:
\item State transition: Invalid $\rightarrow$ Shared

\textbf{Result}:

\begin{itemize}
\item PE1: X = 0 (Shared)
\item PE3: X = 0 (Shared)
\item Memory: X = 0 (updated)
\item All consistent

\subsubsection{Key Points}

\begin{itemize}
\item Coherency maintained throughout
\item Invalidations prevent stale data reads
\item Modified state identifies most recent value
\item Snoop reads fetch from other caches efficiently
\item Write-backs occur when transitioning from Modified to Shared

\subsection{Scalability of UMA Systems}

\subsubsection{Limitation}

\textbf{Challenges}:

\begin{itemize}
\item Bus snooping doesn't scale well
\item Bus contention increases with more processors
\item Snoop bus becomes bottleneck
\item Memory bus also becomes bottleneck

\subsubsection{Practical Limit}

\begin{itemize}
\item Up to ~32 processing elements with bus-based design
\item Not a hard threshold but approximate practical limit
\item Beyond this, contention significantly degrades performance

\subsubsection{Alternative Interconnectsconnects}

\paragraph{Crossbar Switches}

\begin{itemize}
\item Alternative to bus-based architecture
\item Allows multiple simultaneous connections
\item Better than simple bus

\paragraph{Multi-Stage Crossbar Switch Network}

\begin{itemize}
\item Multiple crossbar switches in network topology
\item Increased parallelism in interconnect
\item Can connect multiple memory banks simultaneously
\item Increases scalability

\subsubsection{Improved Scalability}

\begin{itemize}
\item With crossbar networks: Up to ~256 processing elements
\item Still limited but much better than bus-based
\item Trade-off: More complex hardware

\subsection{Non-Uniform Memory Access (NUMA)}

\subsubsection{Designed for Even Higher Scalability}

\textbf{Goals}:

\begin{itemize}
\item Target: Thousands of processing elements
\item Beyond limits of UMA systems
\item Still uses shared memory model
\item Communication through shared address space

\subsubsection{Key Difference from UMA}

\textbf{Non-Uniform Access Times}:

\begin{itemize}
\item Memory access time DEPENDS on which processor is accessing
\item Different processors experience different latencies
\item Memory perspective is non-uniform

\subsubsection{Architecture}

\textbf{Structure}:

\begin{itemize}
\item Each processor has local memory
\item Faster to access local memory
\item Slower to access remote memory (other processors' local memory)
\item But all memory accessible by all processors (shared address space)

\subsubsection{Access Time Difference}

\begin{itemize}
\item Remote memory access: 4-5 times more cycles than local
\item Significant performance impact
\item Programming must consider locality

\subsubsection{Operating System Role}

\textbf{Optimization Responsibilities}:

\begin{itemize}
\item Must use special algorithms for memory optimization
\item Workload distribution affects performance
\item Should relocate memory blocks for optimization
\item Goal: Maximize local accesses, minimize remote accesses
\item Global optimization problem

\subsection{Two Types of NUMA}

\subsubsection{1. NC-NUMA (Non-Cached NUMA)}

![Bus Snooping](img/Multiprocessors_mmu.jpg)

\textbf{Characteristics}:

\begin{itemize}
\item No caches shown in architecture
\item Processors directly access memory
\item Simpler but slower

\subsubsection{2. CC-NUMA (Cache-Coherent NUMA)}

![Bus Snooping](img/Multiprocessors_cmmu.jpg)

\textbf{Characteristics}:

\begin{itemize}
\item Includes caches at each node
\item Must maintain cache coherence
\item More complex but better performance
\item Cannot use bus snooping (not scalable enough)
\item Solution: Directory-based coherence

\subsection{Directory-Based Cache Coherence}

Used in CC-NUMA systems for scalable cache coherence.

\subsubsection{What is Directory?}

\textbf{Definition}:

\begin{itemize}
\item Data structure tracking cache contents
\item Distributed across system
\item Stores information about which blocks are cached where
\item Can be in memory or separate hardware

\subsubsection{Purpose}

\textbf{Functionality}:

\begin{itemize}
\item Cache controllers check directory to find block locations
\item Determines if other caches have copies of block
\item Enables coherence without bus snooping
\item Scalable to thousands of processors

\subsubsection{Organization}

\textbf{Distributed Structure}:

\begin{itemize}
\item Directory can be distributed
\item Each node has local directory
\item Local directory tracks blocks from local memory address range
\item Information about which caches have those blocks
\item Blocks from other address ranges tracked in other directories

\subsubsection{Operation}

\textbf{Access Process}:

\begin{itemize}
\item Cache controller accesses appropriate directory
\item Local directory if accessing local address range
\item Remote directory if accessing remote address range
\item Directory provides information about block locations
\item Can then send invalidations or updates as needed

\subsubsection{Write Policy}

\begin{itemize}
\item Typically use write-through policy

\subsection{Key Takeaways}

\begin{enumerate}
\item Multiprocessors overcome single-processor performance limitations
\item Shared memory provides communication mechanism between processors
\item Cache coherence is essential for correct parallel program execution
\item Bus snooping works well for small-scale systems (up to ~32 processors)
\item MESI protocol is widely adopted for cache coherence
\item UMA systems provide uniform access but limited scalability
\item NUMA systems enable thousands of processors with non-uniform access
\item Directory-based coherence enables scalable cache coherence
\item Operating system plays crucial role in workload balancing and optimization
\end{enumerate}

10. Trade-offs exist between simplicity, performance, and scalability

\subsection{Summary}

Multiprocessor systems have become the standard in modern computing, from smartphones to supercomputers, enabling the parallel processing power required for contemporary applications while managing the complex interactions between multiple processors sharing memory resources.


% % Lecture 20: Storage and Interfacing
% \input{lecture-20}

\end{document}
